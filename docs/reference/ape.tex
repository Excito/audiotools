%This work is licensed under the
%Creative Commons Attribution-Share Alike 3.0 United States License.
%To view a copy of this license, visit
%http://creativecommons.org/licenses/by-sa/3.0/us/ or send a letter to
%Creative Commons,
%171 Second Street, Suite 300,
%San Francisco, California, 94105, USA.

\chapter{Monkey's Audio}
Monkey's Audio is a lossless RIFF WAVE compressor.
Unlike FLAC, which is a PCM compressor, Monkey's Audio also stores
IFF chunks and reproduces the original WAVE file in its entirety rather
than storing only the data it contains.
All of its fields are little-endian.

\section{the Monkey's Audio File Stream}
\begin{figure}[h]
\includegraphics{figures/ape_stream.pdf}
\end{figure}

\section{the Monkey's Audio Descriptor}
\begin{figure}[h]
\includegraphics{figures/ape_descriptor.pdf}
\end{figure}
\par
\noindent
\VAR{Version} is the encoding software's version times 1000.
i.e. Monkey's Audio 3.99 = 3990

\section{the Monkey's Audio header}
\begin{figure}[h]
\includegraphics{figures/ape_header.pdf}
\end{figure}
{\relsize{-2}
\begin{equation}
\text{Length in Seconds} = \frac{((\text{Total Frames} - 1) \times \text{Blocks Per Frame}) + \text{Final Frame Blocks}}{\text{Sample Rate}}
\end{equation}
}
\section{the APEv2 Tag}
\label{apev2}
The APEv2 tag is a little-endian metadata tag appended to
Monkey's Audio files, among others.
\begin{figure}[h]
\includegraphics{figures/apev2_tag.pdf}
\end{figure}
\par
\noindent
\VAR{Item Key} is an ASCII string from the range 0x20 to 0x7E.
\VAR{Item Value} is typically a UTF-8 encoded string, but may
also be binary depending on the Flags.

\begin{multicols}{2}
{\relsize{-2}
\begin{description}
\item[Abstract] Abstract
\item[Album] album name
\item[Artist] performing artist
\item[Bibliography] Bibliography/Discography
\item[Catalog] catalog number
\item[Comment] user comment
\item[Composer] original composer
\item[Conductor] conductor
\item[Copyright] copyright holder
\item[Debut album] debut album name
\item[Dummy] place holder
\item[EAN/UPC] EAN-13/UPC-A bar code identifier
\item[File] file location
\item[Genre] genre
\item[Index] indexes for quick access
\item[Introplay] characteristic part of piece for intro playing
\item[ISBN] ISBN number with check digit
\item[ISRC] International Standard Recording Number
\item[Language] used Language(s) for music/spoken words
\item[LC] Label Code
\item[Media] source media
\item[Publicationright] publication right holder
\item[Publisher] record label or publisher
\item[Record Date] record date
\item[Record Location] record location
\item[Related] location of related information
\item[Subtitle] track subtitle
\item[Title] track title
\item[Track] track number
\item[Year] release date
\end{description}
}
\end{multicols}

\pagebreak

\subsection{the APEv2 Tag Header/Footer}
\begin{figure}[h]
\includegraphics{figures/apev2_tagheader.pdf}
\end{figure}
\par
\noindent
The format of the APEv2 header and footer are identical
except for the \VAR{Is Header} tag.
\VAR{Version} is typically 2000 (stored little-endian).
\VAR{Tag Size} is the size of the entire APEv2 tag, including the
footer but excluding the header.
\VAR{Item Count} is the number of individual tag items.

\subsection{the APEv2 Flags}
\begin{figure}[h]
\includegraphics{figures/apev2_flags.pdf}
\end{figure}
\par
\noindent
This flags field is used by both the APEv2 header/footer and the
individual tag items.
The \VAR{Encoding} field indicates the encoding of its value:

\begin{inparaenum}
\item[\texttt{00} = ] UTF-8,
\item[\texttt{01} = ] Binary,
\item[\texttt{10} = ] External Link,
\item[\texttt{11} = ] Reserved
\end{inparaenum}
.
