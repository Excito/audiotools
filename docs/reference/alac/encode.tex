%This work is licensed under the
%Creative Commons Attribution-Share Alike 3.0 United States License.
%To view a copy of this license, visit
%http://creativecommons.org/licenses/by-sa/3.0/us/ or send a letter to
%Creative Commons,
%171 Second Street, Suite 300,
%San Francisco, California, 94105, USA.

\section{ALAC Encoding}

To encode an ALAC file, we need a stream of PCM sample integers
along with that stream's sample rate, bits-per-sample and number of
channels.
We'll start by encoding all of the non-audio ALAC atoms,
most of which are contained within the \ATOM{moov} atom.
There's over twenty atoms in a typical ALAC file,
most of which are packed with seemingly redundant or
nonessential data,
so it will take awhile before we can move on to the actual
audio encoding process.

Remember, all of an ALAC's fields are big-endian.

%This work is licensed under the
%Creative Commons Attribution-Share Alike 3.0 United States License.
%To view a copy of this license, visit
%http://creativecommons.org/licenses/by-sa/3.0/us/ or send a letter to
%Creative Commons,
%171 Second Street, Suite 300,
%San Francisco, California, 94105, USA.

\subsection{ALAC Atoms}
\begin{wrapfigure}[6]{r}{1.5in}
\includegraphics{alac/figures/atoms.pdf}
\end{wrapfigure}
We'll encode our ALAC file in iTunes order, which means
it contains the \ATOM{ftyp}, \ATOM{moov}, \ATOM{free} and
\ATOM{mdat} atoms, in that order.

\subsubsection{the ftyp Atom}

\begin{table}[h]
\begin{tabular}{|l|r|l|}
\hline
Field & Size & Value \\
\hline
atom length & 32 & 32 \\
atom type & 32 & `ftyp' (\texttt{0x66747970}) \\
\hline
major brand & 32 & `M4A ' (\texttt{0x4d344120}) \\
major brand version & 32 & \texttt{0} \\
compatible brand & 32 & `M4A ' (\texttt{0x4d344120}) \\
compatible brand & 32 & `mp42' (\texttt{0x6d703432}) \\
compatible brand & 32 & `isom' (\texttt{0x69736f6d}) \\
compatible brand & 32 & \texttt{0x00000000} \\
\hline
\end{tabular}
\end{table}

\subsubsection{the moov Atom}

\begin{table}[h]
\begin{tabular}{|l|r|l|}
\hline
Field & Size & Value \\
\hline
atom length & 32 & \ATOM{mvhd} size + \ATOM{trak} size + \ATOM{udta} size + 8 \\
atom type & 32 & `moov' (\texttt{0x6d6f6f76}) \\
\hline
\ATOM{mvhd} atom & \ATOM{mvhd} size & \ATOM{mvhd} data \\
\ATOM{trak} atom & \ATOM{trak} size & \ATOM{trak} data \\
\ATOM{udta} atom & \ATOM{udta} size & \ATOM{udta} data \\
\hline
\end{tabular}
\end{table}

\clearpage

\subsubsection{the mvhd Atom}

\begin{table}[h]
\begin{tabular}{|l|r|l|}
\hline
Field & Size & Value \\
\hline
atom length & 32 & 108/120 \\
atom type & 32 & `mvhd' (\texttt{0x6d766864}) \\
\hline
version & 8 & \texttt{0x00} \\
flags & 24 & \texttt{0x000000} \\
created date & 32/64 & creation date as Mac UTC \\
modified date & 32/64 & modification date as Mac UTC \\
time scale & 32 & sample rate \\
duration & 32/64 & total PCM frames \\
playback speed & 32 & \texttt{0x10000} \\
user volume & 16 & \texttt{0x100} \\
padding & 80 & \texttt{0x00000000000000000000} \\
window geometry matrix a & 32 & \texttt{0x10000} \\
window geometry matrix b & 32 & \texttt{0} \\
window geometry matrix u & 32 & \texttt{0} \\
window geometry matrix c & 32 & \texttt{0} \\
window geometry matrix d & 32 & \texttt{0x10000} \\
window geometry matrix v & 32 & \texttt{0} \\
window geometry matrix x & 32 & \texttt{0} \\
window geometry matrix y & 32 & \texttt{0} \\
window geometry matrix w & 32 & \texttt{0x40000000} \\
QuickTime preview & 64 & \texttt{0} \\
QuickTime still poster & 32 & \texttt{0} \\
QuickTime selection time & 64 & \texttt{0} \\
QuickTime current time & 32 & \texttt{0} \\
next track ID & 32 & \texttt{2} \\
\hline
\end{tabular}
\end{table}

If \VAR{version} is 0, \VAR{created date}, \VAR{modified date} and
\VAR{duration} are 32 bit fields.
Otherwise, they are 64 bit fields.
The \VAR{created date} and \VAR{modified date} are seconds
since the Macintosh Epoch, which is 00:00:00, January 1st, 1904.\footnote{Why 1904?  It's the first leap year of the 20th century.}
To convert a Unix Epoch timestamp (seconds since January 1st, 1970) to
a Macintosh Epoch, one needs to add 24,107 days -
or \texttt{2082844800} seconds.

\clearpage

\subsubsection{the trak Atom}
\begin{tabular}{|l|r|l|}
\hline
Field & Size & Value \\
\hline
atom length & 32 & \ATOM{tkhd} size + \ATOM{mdia} size + 8 \\
atom type & 32 & `trak' (\texttt{0x7472616b}) \\
\hline
\ATOM{tkhd} atom & \ATOM{tkhd} size & \ATOM{tkhd} data \\
\ATOM{mdia} atom & \ATOM{mdia} size & \ATOM{mdia} data \\
\hline
\end{tabular}

\subsubsection{the tkhd Atom}

\begin{table}[h]
\begin{tabular}{|l|r|l|}
\hline
Field & Size & Value \\
\hline
atom length & 32 & 92/104 \\
atom type & 32 & `tkhd' (\texttt{0x746b6864}) \\
\hline
version & 8 & \texttt{0x00} \\
padding & 20 & \texttt{0x000000} \\
track in poster & 1 & \texttt{0} \\
track in preview & 1 & \texttt{1} \\
track in movie & 1 & \texttt{1} \\
track enabled & 1 & \texttt{1} \\
created date & 32/64 & creation date as Mac UTC \\
modified date & 32/64 & modification date as Mac UTC \\
track ID & 32 & \texttt{1} \\
padding & 32 & \texttt{0x00000000} \\
duration & 32/64 & total PCM frames \\
padding & 64 & \texttt{0x0000000000000000} \\
video layer & 16 & \texttt{0} \\
QuickTime alternate & 16 & \texttt{0} \\
volume & 16 & \texttt{0x1000} \\
padding & 16 & \texttt{0x0000} \\
video geometry matrix a & 32 & \texttt{0x10000} \\
video geometry matrix b & 32 & \texttt{0} \\
video geometry matrix u & 32 & \texttt{0} \\
video geometry matrix c & 32 & \texttt{0} \\
video geometry matrix d & 32 & \texttt{0x10000} \\
video geometry matrix v & 32 & \texttt{0} \\
video geometry matrix x & 32 & \texttt{0} \\
video geometry matrix y & 32 & \texttt{0} \\
video geometry matrix w & 32 & \texttt{0x40000000} \\
video width & 32 & \texttt{0} \\
video height & 32 & \texttt{0} \\
\hline
\end{tabular}
\end{table}

\clearpage

\subsubsection{the mdia Atom}

\begin{table}[h]
\begin{tabular}{|l|r|l|}
\hline
Field & Size & Value \\
\hline
atom length & 32 & \ATOM{mdhd} size + \ATOM{hdlr} size + \ATOM{minf} size + 8 \\
atom type & 32 & `mdia' (\texttt{0x6d646961}) \\
\hline
\ATOM{mdhd} atom & \ATOM{mdhd} size & \ATOM{mdhd} data \\
\ATOM{hdlr} atom & \ATOM{hdlr} size & \ATOM{hdlr} data \\
\ATOM{minf} atom & \ATOM{minf} size & \ATOM{minf} data \\
\hline
\end{tabular}
\end{table}

\subsubsection{the mdhd Atom}

\begin{table}[h]
\begin{tabular}{|l|r|l|}
\hline
Field & Size & Value \\
\hline
atom length & 32 & 32/44 \\
atom type & 32 & `mdhd' (\texttt{0x6d646864}) \\
\hline
version & 8 & \texttt{0x00} \\
flags & 24 & \texttt{0x000000} \\
created date & 32/64 & creation date as Mac UTC \\
modified date & 32/64 & modification date as Mac UTC \\
time scale & 32 & sample rate \\
duration & 32/64 & total PCM frames \\
padding & 1 & \texttt{0} \\
language & 5 & \\
language & 5 & language value as ISO 639-2 \\
language & 5 & \\
QuickTime quality & 16 & \texttt{0} \\
\hline
\end{tabular}
\end{table}
Note the three, 5-bit \VAR{language} fields.
By adding 0x60 to each value and converting the result to ASCII characters,
the result is an \href{http://www.loc.gov/standards/iso639-2/}{ISO 639-2}
string of the file's language representation.
For example, given the values \texttt{0x15}, \texttt{0x0E} and \texttt{0x04}:
\begin{align*}
\text{language}_0 &= \texttt{0x15} + \texttt{0x60} = \texttt{0x75} = \texttt{u} \\
\text{language}_1 &= \texttt{0x0E} + \texttt{0x60} = \texttt{0x6E} = \texttt{n} \\
\text{language}_2 &= \texttt{0x04} + \texttt{0x60} = \texttt{0x64} = \texttt{d}
\end{align*}
Which is the code `\texttt{und}', meaning `undetermined' - which is typical.

\clearpage

\subsubsection{the hdlr Atom}
\label{alac_hdlr}
\begin{tabular}{|l|r|l|}
\hline
Field & Size & Value \\
\hline
atom length & 32 & 33 + component \\
atom type & 32 & `hdlr' (\texttt{0x68646c72}) \\
\hline
version & 8 & \texttt{0x00} \\
flags & 24 & \texttt{0x000000} \\
QuickTime type & 32 & \texttt{0x00000000} \\
QuickTime subtype & 32 & `soun' (\texttt{0x736f756e}) \\
QuickTime manufacturer & 32 & \texttt{0x00000000} \\
QuickTime component reserved flags & 32 & \texttt{0x00000000} \\
QuickTime component reserved flags mask & 32 & \texttt{0x00000000} \\
component name length & 8 & \texttt{0x00} \\
component name & component name length $\times$ 8 & \\
\hline
\end{tabular}


\subsubsection{the minf Atom}
\begin{tabular}{|l|r|l|}
\hline
Field & Size & Value \\
\hline
atom length & 32 & \ATOM{smhd} size + \ATOM{dinf} size + \ATOM{stbl} size + 8 \\
atom type & 32 & `minf' (\texttt{0x6d696e66}) \\
\hline
\ATOM{smhd} atom & \ATOM{smhd} size & \ATOM{smhd} data \\
\ATOM{dinf} atom & \ATOM{dinf} size & \ATOM{dinf} data \\
\ATOM{stbl} atom & \ATOM{stbl} size & \ATOM{stbl} data \\
\hline
\end{tabular}

\subsubsection{the smhd Atom}
\begin{tabular}{|l|r|l|}
\hline
Field & Size & Value \\
\hline
atom length & 32 & 16 \\
atom type & 32 & `smhd' (\texttt{0x736d6864}) \\
\hline
version & 8 & \texttt{0x00} \\
flags & 24 & \texttt{0x000000} \\
audio balance & 16 & \texttt{0x0000} \\
padding & 16 & \texttt{0x0000} \\
\hline
\end{tabular}

\subsubsection{the dinf Atom}
\begin{tabular}{|l|r|l|}
\hline
Field & Size & Value \\
\hline
atom length & 32 & \ATOM{dref} size + 8 \\
atom type & 32 & `dinf' (\texttt{0x64696e66}) \\
\hline
\ATOM{dref} atom & \ATOM{dref} size & \ATOM{dref} data \\
\hline
\end{tabular}

\clearpage

\subsubsection{the dref Atom}

\begin{table}[h]
\begin{tabular}{|l|r|l|}
\hline
Field & Size & Value \\
\hline
atom length & 32 & 28 \\
atom type & 32 & `dref' (\texttt{0x64726566}) \\
\hline
version & 8 & \texttt{0x00} \\
flags & 24 & \texttt{0x000000} \\
number of references & 32 & \texttt{1} \\
\hline
\hline
reference atom size & 32 & \texttt{12} \\
reference atom type & 32 & `url ' (\texttt{0x75726c20}) \\
reference atom data & 32 & \texttt{0x00000001} \\
\hline
\end{tabular}
\end{table}

\subsubsection{the stbl Atom}

\begin{table}[h]
\begin{tabular}{|l|r|l|}
\hline
Field & Size & Value \\
\hline
atom length & 32 & \ATOM{stsd} size + \ATOM{stts} size + \ATOM{stsc} size + \\
& & \ATOM{stsz} size + \ATOM{stco} size + 8 \\
atom type & 32 & `stbl' (\texttt{0x7374626c}) \\
\hline
\ATOM{stsd} atom & \ATOM{stsd} size & \ATOM{stsd} data \\
\ATOM{stts} atom & \ATOM{stts} size & \ATOM{stts} data \\
\ATOM{stsc} atom & \ATOM{stsc} size & \ATOM{stsc} data \\
\ATOM{stsz} atom & \ATOM{stsz} size & \ATOM{stsz} data \\
\ATOM{stco} atom & \ATOM{stco} size & \ATOM{stco} data \\
\hline
\end{tabular}
\end{table}

\subsubsection{the stsd Atom}

\begin{table}[h]
\begin{tabular}{|l|r|l|}
\hline
Field & Size & Value \\
\hline
atom length & 32 & \ATOM{alac} size + 16 \\
atom type & 32 & `stsd' (\texttt{0x73747364}) \\
\hline
version & 8 & \texttt{0x00} \\
flags & 24 & \texttt{0x000000} \\
number of descriptions & 32 & \texttt{1} \\
\hline
\ATOM{alac} atom & \ATOM{alac} size & \ATOM{alac} data \\
\hline
\end{tabular}
\end{table}

\clearpage

\subsubsection{the alac Atom}

\begin{table}[h]
\begin{tabular}{|l|r|l|}
\hline
Field & Size & Value \\
\hline
atom length & 32 & 72 \\
atom type & 32 & `alac' (\texttt{0x616c6163}) \\
\hline
reserved & 48 & \texttt{0x000000000000} \\
reference index & 16 & \texttt{1} \\
version & 16 & \texttt{0} \\
revision level & 16 & \texttt{0} \\
vendor & 32 & \texttt{0x00000000} \\
channels & 16 & channel count \\
bits per sample & 16 & bits per sample \\
compression ID & 16 & \texttt{0} \\
audio packet size & 16 & \texttt{0} \\
sample rate & 32 & \texttt{0xAC440000} \\
\hline
\hline
atom length & 32 & 36 \\
atom type & 32 & `alac' (\texttt{0x616c6163}) \\
\hline
padding & 32 & \texttt{0x00000000} \\
max samples per frame & 32 & largest number of PCM frames per ALAC frame \\
padding & 8 & \texttt{0x00} \\
sample size & 8 & bits per sample \\
history multiplier & 8 & \texttt{40} \\
initial history & 8 & \texttt{10} \\
maximum K & 8 & \texttt{14} \\
channels & 8 & channel count \\
unknown & 16 & \texttt{0x00FF} \\
max coded frame size & 32 & largest ALAC frame size, in bytes \\
bitrate & 32 & $((\text{\ATOM{mdat} size} \times 8 ) \div (\text{total PCM frames} \div \text{sample rate}))$ \\
sample rate & 32 & sample rate \\
\hline
\end{tabular}
\end{table}
The \VAR{history multiplier}, \VAR{initial history} and \VAR{maximum K}
values are encode-time options, typically set to 40, 10 and 14,
respectively.

Note that the \VAR{bitrate} field can't be known in advance;
we must fill that value with 0 for now and then
return to this atom once encoding is completed
and its size has been determined.

\clearpage

\subsubsection{the stts Atom}

\begin{table}[h]
\begin{tabular}{|l|r|l|}
\hline
Field & Size & Value \\
\hline
atom length & 32 & number of times $\times$ 8 + 16\\
atom type & 32 & `stts' (\texttt{0x73747473}) \\
\hline
version & 8 & \texttt{0x00} \\
flags & 24 & \texttt{0x000000} \\
number of times & 32 & \\
\hline
frame count 1 & 32 & number of occurrences \\
frame duration 1 & 32 & PCM frame count \\
\hline
\multicolumn{3}{|c|}{...} \\
\hline
\end{tabular}
\end{table}
This atom keeps track of how many different sizes of ALAC frames
occur in the ALAC file, in PCM frames.
It will typically have only two ``times'', the block size we're
using for most of our samples and the final block size for
any remaining samples.

For example, let's imagine encoding a 1 minute audio file
at 44100Hz with a block size of 4096 frames.
This file has a total of 2,646,000 PCM frames ($60 \times 44100 = 2646000$).
2,646,000 PCM frames divided by a 4096 block size means
we have 645 ALAC frames of size 4096, and 1 ALAC frame of size 4080.

Therefore:
\begin{align*}
\text{number of times} &= 2 \\
\text{frame count}_1 &= 645 \\
\text{frame duration}_1 &= 4096 \\
\text{frame count}_2 &= 1 \\
\text{frame duration}_2 &= 4080
\end{align*}

\subsubsection{the stsc Atom}

\begin{table}[h]
\begin{tabular}{|l|r|l|}
\hline
Field & Size & Value \\
\hline
atom length & 32 & entries $\times$ 12 + 16 \\
atom type & 32 & `stsc' (\texttt{0x73747363}) \\
\hline
version & 8 & \texttt{0x00} \\
flags & 24 & \texttt{0x000000} \\
number of entries & 32 & \\
\hline
first chunk & 32 & \\
ALAC frames per chunk & 32 & \\
description index & 32 & \texttt{1} \\
\hline
\multicolumn{3}{|c|}{...} \\
\hline
\end{tabular}
\end{table}

This atom stores how many ALAC frames are in a given ``chunk''.
In this instance a ``chunk'' represents an entry in
the \ATOM{stco} atom table, used for seeking backwards and forwards
through the file.
\VAR{First chunk} is the starting offset of its frames-per-chunk
value, beginning at 1.

As an example, let's take a one minute, 44100Hz audio file
that's been broken into 130 chunks
(each with an entry in the \ATOM{stco} atom).
Its \ATOM{stsc} entries would typically be:
\begin{align*}
\text{first chunk}_1 &= 1 \\
\text{frames per chunk}_1 &= 5 \\
\text{first chunk}_2 &= 130 \\
\text{frames per chunk}_2 &= 1
\end{align*}
What this means is that chunks 1 through 129 have 5 ALAC frames each
while chunk 130 has 1 ALAC frame.
This is a total of 646 ALAC frames, which matches the contents of
the \ATOM{stts} atom.

\subsubsection{the stsz Atom}

\begin{tabular}{|l|r|l|}
\hline
Field & Size & Value \\
\hline
atom length & 32 & sizes $\times$ 4 + 20 \\
atom type & 32 & `stsz' (\texttt{0x7374737a}) \\
\hline
version & 8 & \texttt{0x00} \\
flags & 24 & \texttt{0x000000} \\
block byte size & 32 & \texttt{0x00000000} \\
number of sizes & 32 & \\
\hline
frame size & 32 & \\
\hline
\multicolumn{3}{|c|}{...} \\
\hline
\end{tabular}

This atom is a list of ALAC frame sizes, each in bytes.
For example, our 646 frame file would have 646 corresponding
\ATOM{stsz} entries.

\subsubsection{the stco Atom}

\begin{tabular}{|l|r|l|}
\hline
Field & Size & Value \\
\hline
atom length & 32 & offset $\times$ 4 + 16 \\
atom type & 32 & `stco' (\texttt{0x7374636f}) \\
\hline
version & 8 & \texttt{0x00} \\
flags & 24 & \texttt{0x000000} \\
number of offsets & 32 & \\
\hline
frame offset & 32 & \\
\hline
\multicolumn{3}{|c|}{...} \\
\hline
\end{tabular}

This atom is a list of absolute file offsets for each chunk, where
each chunk is typically 5 ALAC frames large.

\clearpage

\subsubsection{the udta Atom}

\begin{tabular}{|l|r|l|}
\hline
Field & Size & Value \\
\hline
atom length & 32 & \ATOM{meta} size + 8 \\
atom type & 32 & `udta' (\texttt{0x75647461}) \\
\hline
\ATOM{meta} atom & \ATOM{meta} size & \ATOM{meta} data \\
\hline
\end{tabular}

\subsubsection{the meta Atom}

\begin{tabular}{|l|r|l|}
\hline
Field & Size & Value \\
\hline
atom length & 32 & \ATOM{hdlr} size + \ATOM{ilst} size + \ATOM{free} size + 12 \\
atom type & 32 & `meta' (\texttt{0x6d657461}) \\
\hline
version & 8 & \texttt{0x00} \\
flags & 24 & \texttt{0x000000} \\
\hline
\ATOM{hdlr} atom & \ATOM{hdlr} size & \ATOM{hdlr} data \\
\ATOM{ilst} atom & \ATOM{ilst} size & \ATOM{ilst} data \\
\ATOM{free} atom & \ATOM{free} size & \ATOM{free} data \\
\hline
\end{tabular}

\subsubsection{the hdlr atom (revisited)}

\begin{tabular}{|l|r|l|}
\hline
Field & Size & Value \\
\hline
atom length & 32 & 34 \\
atom type & 32 & `hdlr' (\texttt{0x68646c72}) \\
\hline
version & 8 & \texttt{0x00} \\
flags & 24 & \texttt{0x000000} \\
QuickTime type & 32 & \texttt{0x00000000} \\
QuickTime subtype & 32 & `mdir' (\texttt{0x6d646972}) \\
QuickTime manufacturer & 32 & `appl' (\texttt{0x6170706c}) \\
QuickTime component reserved flags & 32 & \texttt{0x00000000} \\
QuickTime component reserved flags mask & 32 & \texttt{0x00000000} \\
component name length & 8 & \texttt{0x00} \\
component name & 0 & \\
\hline
\end{tabular}

This atom is laid out identically to the ALAC file's primary
\ATOM{hdlr} atom (described on page \pageref{alac_hdlr}).
The only difference is the contents of its fields.

\subsubsection{the ilst Atom}

This atom is a collection of \ATOM{data} sub-atoms
and is described on page \pageref{m4a_meta}.

\subsubsection{the free Atom}

These atoms are simple collection of NULL bytes which can easily be
resized to make room for other atoms without rewriting the entire file.


\clearpage

\subsection{Encoding mdat Atom}
\ALGORITHM{PCM frames, various encoding parameters:
\newline
\begin{tabular}{rl}
parameter & typical value \\
\hline
block size & 4096 \\
initial history & 40 \\
history multiplier & 10 \\
maximum K & 14 \\
interlacing shift & 2 \\
minimum interlacing leftweight & 0 \\
maximum interlacing leftweight & 4 \\
\end{tabular}
}{an encoded \texttt{mdat} atom}
\SetKwData{BLOCKSIZE}{block size}
$0 \rightarrow$ \WRITE 32 unsigned bits\tcc*[r]{placeholder length}
$\texttt{"mdat"} \rightarrow$ \WRITE 4 bytes\;
\While{PCM frames remain}{
  take \BLOCKSIZE PCM frames from the input\;
  \hyperref[alac:encode_frameset]{write PCM frames to frameset}\;
}
return to start of \texttt{mdat} atom and write actual length\;
\EALGORITHM
\begin{figure}[h]
\includegraphics{alac/figures/stream.pdf}
\end{figure}

\clearpage

\subsection{Encoding Frameset}
\label{alac:encode_frameset}
{\relsize{-2}
\ALGORITHM{1 or more channels of PCM frames}{1 or more ALAC frames as a frameset}
\SetKwData{CHANCOUNT}{channel count}
\SetKwData{FRAMEDATA}{frame channels}
\Switch{\CHANCOUNT}{
  \uCase{1}{
    \hyperref[alac:encode_frame]{encode mono as 1 channel frame}\;
  }
  \uCase{2}{
    \hyperref[alac:encode_frame]{encode left,right as 2 channel frame}\;
  }
  \uCase{3}{
    \hyperref[alac:encode_frame]{encode center as 1 channel frame}\;
    \hyperref[alac:encode_frame]{encode left,right as 2 channel frame}\;
  }
  \uCase{4}{
    \hyperref[alac:encode_frame]{encode center as 1 channel frame}\;
    \hyperref[alac:encode_frame]{encode left,right as 2 channel frame}\;
    \hyperref[alac:encode_frame]{encode center surround as 1 channel frame}\;
  }
  \uCase{5}{
    \hyperref[alac:encode_frame]{encode center as 1 channel frame}\;
    \hyperref[alac:encode_frame]{encode left,right as 2 channel frame}\;
    \hyperref[alac:encode_frame]{encode left surround,right surround as 2 channel frame}\;
  }
  \uCase{6}{
    \hyperref[alac:encode_frame]{encode center as 1 channel frame}\;
    \hyperref[alac:encode_frame]{encode left,right as 2 channel frame}\;
    \hyperref[alac:encode_frame]{encode left surround,right surround as 2 channel frame}\;
    \hyperref[alac:encode_frame]{encode LFE as 1 channel frame}\;
  }
  \uCase{7}{
    \hyperref[alac:encode_frame]{encode center as 1 channel frame}\;
    \hyperref[alac:encode_frame]{encode left,right as 2 channel frame}\;
    \hyperref[alac:encode_frame]{encode left surround,right surround as 2 channel frame}\;
    \hyperref[alac:encode_frame]{encode center surround as 1 channel frame}\;
    \hyperref[alac:encode_frame]{encode LFE as 1 channel frame}\;
  }
  \Case{8}{
    \hyperref[alac:encode_frame]{encode center as 1 channel frame}\;
    \hyperref[alac:encode_frame]{encode left center,right center as 2 channel frame}\;
    \hyperref[alac:encode_frame]{encode left,right as 2 channel frame}\;
    \hyperref[alac:encode_frame]{encode left surround,right surround as 1 channel frame}\;
    \hyperref[alac:encode_frame]{encode LFE as 1 channel frame}\;
  }
  $7 \rightarrow$ \WRITE 3 unsigned bits\;
  byte align output stream\;
}
\Return encoded frameset\;
\EALGORITHM
}

\subsubsection{Channel Assignment}
\begin{tabular}{r|l}
channels & assignment \\
\hline
1 & mono \\
2 & left, right \\
3 & center, left, right \\
4 & center, left, right, center surround \\
5 & center, left, right, left surround, right surround \\
6 & center, left, right, left surround, right surround, LFE \\
7 & center, left, right, left surround, right surround, center surround, LFE \\
8 & center, left center, right center, left, right, left surround, right surround, LFE \\
\end{tabular}

\clearpage

\subsection{Encoding Frame}
\label{alac:encode_frame}
{\relsize{-1}
\ALGORITHM{1 or 2 channels of PCM data, encoding parameters}{a compressed or uncompressed ALAC frame}
\SetKwData{PCMCOUNT}{PCM frame count}
\SetKwData{CHANCOUNT}{channel count}
\SetKwData{COMPRESSED}{compressed frame}
\SetKwData{UNCOMPRESSED}{uncompressed frame}
\SetKwData{BPS}{bits per sample}
\SetKwFunction{LEN}{len}
$\CHANCOUNT - 1 \rightarrow$ \WRITE 3 unsigned bits\;
\eIf{$\text{\PCMCOUNT} \geq 10$}{
  $\UNCOMPRESSED \leftarrow$ \hyperref[alac:write_uncompressed_frame]{encode channel as uncompressed frame}\;
  $\COMPRESSED \leftarrow$ \hyperref[alac:write_compressed_frame]{encode channels as compressed frame}\;
  \uIf{residual overflow occurred}{
    \Return \UNCOMPRESSED\;
  }
  \uElseIf{$\LEN(\COMPRESSED) \geq \LEN(\UNCOMPRESSED)$}{
    \Return \UNCOMPRESSED\;
  }
  \Else{
    \Return \COMPRESSED\;
  }
}{
  \Return \UNCOMPRESSED\;
}
\EALGORITHM
}

\subsection{Encoding Uncompressed Frame}
\label{alac:write_uncompressed_frame}
{\relsize{-1}
\ALGORITHM{1 or 2 channels of PCM data, encoding parameters}{an uncompressed ALAC frame}
\SetKwData{PCMCOUNT}{PCM frame count}
\SetKwData{CHANCOUNT}{channel count}
\SetKwData{BLOCKSIZE}{block size}
\SetKwData{BPS}{bits per sample}
\SetKwData{CHANNEL}{channel}
$0 \rightarrow$ \WRITE 16 unsigned bits\tcc*[r]{unused}
\eIf{$\text{\PCMCOUNT} = \text{encoding parameter's \BLOCKSIZE}$}{
  $0 \rightarrow$ \WRITE 1 unsigned bit\;
}{
  $1 \rightarrow$ \WRITE 1 unsigned bit\;
}
$0 \rightarrow$ \WRITE 2 unsigned bits\tcc*[r]{uncompressed LSBs}
$1 \rightarrow$ \WRITE 1 unsigned bit\tcc*[r]{not compressed}
\If{$\text{\PCMCOUNT} \neq \text{encoding parameter's \BLOCKSIZE}$}{
  $\PCMCOUNT \rightarrow$ \WRITE 32 unsigned bits\;
}
\For{$i \leftarrow 0$ \emph{\KwTo}\PCMCOUNT}{
  \For{$c \leftarrow 0$ \emph{\KwTo}\CHANCOUNT}{
    $\text{\CHANNEL}_{c~i} \rightarrow$ \WRITE (\BPS) signed bits\;
  }
}
\Return uncompressed frame\;
\EALGORITHM
}

\begin{figure}[h]
  \includegraphics{alac/figures/uncompressed_frame.pdf}
\end{figure}

\clearpage

\subsection{Encoding Compressed Frame}
\label{alac:write_compressed_frame}
{\relsize{-1}
\ALGORITHM{1 or 2 channels of PCM data, encoding parameters}{a compressed ALAC frame, or a \textit{residual overflow} exception}
\SetKwData{PCMCOUNT}{PCM frame count}
\SetKwData{CHANCOUNT}{channel count}
\SetKwData{CHANNEL}{channel}
\SetKwData{SHIFTED}{shifted}
\SetKwData{FRAME}{frame}
\SetKwData{BPS}{bits per sample}
\SetKwData{HASLSBS}{uncompressed LSBs}
\SetKwData{ISHIFT}{interlacing shift}
\SetKwData{MINWEIGHT}{minimum leftweight}
\SetKwData{MAXWEIGHT}{maximum leftweight}
\SetKwData{LSB}{LSB}
\eIf{$\text{\BPS} \leq 16$}{
  \HASLSBS $\leftarrow 0$\;
  \LSB $\leftarrow$ \texttt{[]}\;
  \For{$c \leftarrow 0$ \emph{\KwTo}\CHANCOUNT}{
    $\text{\SHIFTED}_c \leftarrow \text{\CHANNEL}_c$\;
  }
}(\tcc*[f]{extract uncompressed LSBs}){
  \HASLSBS $\leftarrow (\text{\BPS} - 16) \div 8$\;
  \For{$i \leftarrow 0$ \emph{\KwTo}\PCMCOUNT}{
    \For{$c \leftarrow 0$ \emph{\KwTo}\CHANCOUNT}{
      $\text{\LSB}_{(i \times \CHANCOUNT) + c} \leftarrow (\text{\CHANNEL}_{c~i}) \bmod~2^{\text{\BPS} - 16}$\;
      $\text{\SHIFTED}_{c~i} \leftarrow \left\lfloor(\text{\CHANNEL}_{c~i}) \div 2^{\text{\BPS} - 16}\right\rfloor$\;
    }
  }
}
\eIf{$\text{\CHANCOUNT} = 1$}{
  \Return \hyperref[alac:write_non_interlaced_frame]{non-interlaced frame}
  $\left\lbrace\begin{tabular}{l}
  $\text{\SHIFTED}_0$ \\
  \HASLSBS \\
  \LSB \\
  \end{tabular}\right.$\;
}{
  \tcc{minimum/maximum leftweight, interlacing shift from encoding parameters}
  \For{l $\leftarrow \MINWEIGHT$ \emph{\KwTo}$\text{\MAXWEIGHT} + 1$}{
    $\text{\FRAME}_l \leftarrow$
    \hyperref[alac:write_interlaced_frame]{interlaced frame}
    $\left\lbrace\begin{tabular}{l}
    $\text{\SHIFTED}_0$ \\
    $\text{\SHIFTED}_1$ \\
    \ISHIFT \\
    leftweight $l$ \\
    \HASLSBS \\
    \LSB \\
    \end{tabular}\right.$\;
  }
  \Return smallest $\text{\FRAME}_l$\;
}
\EALGORITHM
}

\clearpage

\subsection{Encoding Non-Interlaced Frame}
\label{alac:write_non_interlaced_frame}
{\relsize{-1}
\ALGORITHM{1 channel of PCM data, uncompressed LSBs, encoding parameters}{a compressed ALAC frame, or a \textit{residual overflow} exception}
\SetKwData{PCMCOUNT}{PCM frame count}
\SetKwData{CHANCOUNT}{channel count}
\SetKwData{BLOCKSIZE}{block size}
\SetKwData{BPS}{bits per sample}
\SetKwData{SAMPLESIZE}{sample size}
\SetKwData{QLPCOEFF}{QLP coefficient}
\SetKwData{RESIDUAL}{residual}
\SetKwData{CHANNEL}{channel}
\SetKwData{UNCOMPRESSEDLSB}{uncompressed LSBs}
\SetKwData{LSB}{LSB}
$0 \rightarrow$ \WRITE 16 unsigned bits\tcc*[r]{unused}
\eIf{$\text{\PCMCOUNT} \neq \text{encoding parameter's \BLOCKSIZE}$}{
  $1 \rightarrow$ \WRITE 1 unsigned bit\;
}{
  $0 \rightarrow$ \WRITE 1 unsigned bit\;
}
$\text{uncompressed LSBs} \rightarrow$ \WRITE 2 unsigned bits\;
$0 \rightarrow$ \WRITE 1 unsigned bit\tcc*[r]{is compressed}
\If{$\text{\PCMCOUNT} \neq \text{encoding parameter's \BLOCKSIZE}$}{
  $\PCMCOUNT \rightarrow$ \WRITE 32 unsigned bits\;
}
$0 \rightarrow$ \WRITE 8 unsigned bits\tcc*[r]{interlacing shift}
$0 \rightarrow$ \WRITE 8 unsigned bits\tcc*[r]{interlacing leftweight}
\SAMPLESIZE $\leftarrow \text{\BPS} - (\text{uncompressed LSBs} \times 8)$\;
$\left.\begin{tabular}{r}
$\text{\QLPCOEFF}_0$ \\
$\text{\RESIDUAL}_0$ \\
\end{tabular}\right\rbrace \leftarrow$
\hyperref[alac:compute_qlp_coeffs]{compute QLP coefficient}
$\left\lbrace\begin{tabular}{l}
$\text{\CHANNEL}_0$ \\
\SAMPLESIZE \\
\end{tabular}\right.$\;
\hyperref[alac:write_subframe_header]{write subframe header with $\text{\QLPCOEFF}_0$}\;
\If{$\text{\UNCOMPRESSEDLSB} > 0$}{
  \For{$i \leftarrow 0$ \emph{\KwTo}\PCMCOUNT}{
    $\text{\LSB}_i \rightarrow$ \WRITE $(\text{\UNCOMPRESSEDLSB} \times 8)$ unsigned bits\;
  }
}
\hyperref[alac:write_residuals]{write residual block $\text{\RESIDUAL}_0$}\;
\BlankLine
\Return non-interlaced ALAC frame\;
\EALGORITHM
}

\begin{figure}[h]
  \includegraphics{alac/figures/noninterlaced_frame.pdf}
\end{figure}

\clearpage

\subsection{Encoding Interlaced Frame}
\label{alac:write_interlaced_frame}
{\relsize{-1}
\ALGORITHM{2 channels of PCM data, interlacing shift, interlacing leftweight, uncompressed LSBs, encoding parameters}{a compressed ALAC frame, or a \textit{residual overflow} exception}
\SetKwData{PCMCOUNT}{PCM frame count}
\SetKwData{CHANCOUNT}{channel count}
\SetKwData{BLOCKSIZE}{block size}
\SetKwData{BPS}{bits per sample}
\SetKwData{SAMPLESIZE}{sample size}
\SetKwData{UNCOMPRESSEDLSB}{uncompressed LSBs}
\SetKwData{INTERLACINGSHIFT}{interlacing shift}
\SetKwData{INTERLACINGLEFTWEIGHT}{interlacing leftweight}
\SetKwData{CHANNEL}{channel}
\SetKwData{CORRELATED}{correlated}
\SetKwData{QLPCOEFF}{QLP coefficient}
\SetKwData{RESIDUAL}{residual}
$0 \rightarrow$ \WRITE 16 unsigned bits\tcc*[r]{unused}
\eIf{$\text{\PCMCOUNT} \neq \text{encoding parameter's \BLOCKSIZE}$}{
  $1 \rightarrow$ \WRITE 1 unsigned bit\;
}{
  $0 \rightarrow$ \WRITE 1 unsigned bit\;
}
$\text{\UNCOMPRESSEDLSB} \rightarrow$ \WRITE 2 unsigned bits\;
$0 \rightarrow$ \WRITE 1 unsigned bit\tcc*[r]{is compressed}
\If{$\text{\PCMCOUNT} \neq \text{encoding parameter's \BLOCKSIZE}$}{
  $\text{\PCMCOUNT} \rightarrow$ \WRITE 32 unsigned bits\;
}
$\text{\INTERLACINGSHIFT} \rightarrow$ \WRITE 8 unsigned bits\;
$\text{\INTERLACINGLEFTWEIGHT} \rightarrow$ \WRITE 8 unsigned bits\;
\BlankLine
$\left.\begin{tabular}{r}
$\text{\CORRELATED}_0$ \\
$\text{\CORRELATED}_1$ \\
\end{tabular}\right\rbrace \leftarrow$
\hyperref[alac:correlate_channels]{correlate channels}
$\left\lbrace\begin{tabular}{l}
$\text{\CHANNEL}_0$ \\
$\text{\CHANNEL}_1$ \\
\INTERLACINGSHIFT \\
\INTERLACINGLEFTWEIGHT \\
\end{tabular}\right.$\;
\BlankLine
\SAMPLESIZE $\leftarrow \text{\BPS} - (\text{uncompressed LSBs} \times 8) + 1$\;
$\left.\begin{tabular}{r}
$\text{\QLPCOEFF}_0$ \\
$\text{\RESIDUAL}_0$ \\
\end{tabular}\right\rbrace \leftarrow$
\hyperref[alac:compute_qlp_coeffs]{compute QLP coefficient}
$\left\lbrace\begin{tabular}{l}
$\text{\CORRELATED}_0$ \\
\SAMPLESIZE \\
\end{tabular}\right.$\;
$\left.\begin{tabular}{r}
$\text{\QLPCOEFF}_1$ \\
$\text{\RESIDUAL}_1$ \\
\end{tabular}\right\rbrace \leftarrow$
\hyperref[alac:compute_qlp_coeffs]{compute QLP coefficient}
$\left\lbrace\begin{tabular}{l}
$\text{\CORRELATED}_1$ \\
\SAMPLESIZE \\
\end{tabular}\right.$\;
\hyperref[alac:write_subframe_header]{write subframe header with $\text{\QLPCOEFF}_0$}\;
\hyperref[alac:write_subframe_header]{write subframe header with $\text{\QLPCOEFF}_1$}\;
\BlankLine
\If{$\text{\UNCOMPRESSEDLSB} > 0$}{
  \For{$i \leftarrow 0$ \emph{\KwTo}\PCMCOUNT}{
    $\text{LSB}_i \rightarrow$ \WRITE $(\text{\UNCOMPRESSEDLSB} \times 8)$ unsigned bits\;
  }
}
\BlankLine
\hyperref[alac:write_residuals]{write residual block $\text{\RESIDUAL}_0$}\;
\hyperref[alac:write_residuals]{write residual block $\text{\RESIDUAL}_1$}\;
\BlankLine
\Return interlaced ALAC frame\;
\EALGORITHM
}

\clearpage

\begin{figure}[h]
  \includegraphics{alac/figures/interlaced_frame.pdf}
\end{figure}

\subsubsection{Correlating Channels}
\label{alac:correlate_channels}
{\relsize{-1}
\ALGORITHM{2 channels of PCM data, interlacing shift, interlacing leftweight}{2 correlated channels of PCM data}
\SetKwData{PCMCOUNT}{PCM frame count}
\SetKwData{LEFTWEIGHT}{interlacing leftweight}
\SetKwData{SHIFT}{interlacing shift}
\SetKwData{CORRELATED}{correlated}
\SetKwData{CHANNEL}{channel}
\eIf{$\text{\LEFTWEIGHT} > 0$}{
  \For{$i \leftarrow 0$ \emph{\KwTo}\PCMCOUNT}{
    $\text{\CORRELATED}_{0~i} \leftarrow \text{\CHANNEL}_{1~i} + \left\lfloor\frac{(\text{\CHANNEL}_{0~i} - \text{\CHANNEL}_{1~i}) \times \text{\LEFTWEIGHT}}{2 ^ \text{\SHIFT}}\right\rfloor$\;
    $\text{\CORRELATED}_{1~i} \leftarrow \text{\CHANNEL}_{0~i} - \text{\CHANNEL}_{1~i}$\;
  }
}{
  \For{$i \leftarrow 0$ \emph{\KwTo}\PCMCOUNT}{
    $\text{\CORRELATED}_{0~i} \leftarrow \text{\CHANNEL}_{0~i}$\;
    $\text{\CORRELATED}_{1~i} \leftarrow \text{\CHANNEL}_{1~i}$\;
  }
}
\Return $\left\lbrace\begin{tabular}{l}
$\text{\CORRELATED}_0$ \\
$\text{\CORRELATED}_1$ \\
\end{tabular}\right.$
\EALGORITHM
\par
\noindent
For example, given an \VAR{interlacing shift} value of 2 and an
\VAR{interlacing leftweight} value of 3:
\par
\noindent
{\relsize{-1}
\begin{tabular}{r||r|r||>{$}r<{$}|>{$}r<{$}|}
$i$ & $\textsf{channel}_{0~i}$ & $\textsf{channel}_{1~i}$ & \textsf{correlated}_{0~i} & \textsf{correlated}_{1~i} \\
\hline
0 & 18 & 2 & 2 + \lfloor((18 - 2) \times 3) \div 2 ^ 2\rfloor = 14 & 18 - 2 = 16 \\
1 & 20 & 3 & 3 + \lfloor((20 - 3) \times 3) \div 2 ^ 2\rfloor = 15 & 20 - 3 = 17 \\
2 & 26 & 0 & 0 + \lfloor((26 - 0) \times 3) \div 2 ^ 2\rfloor = 19 & 26 - 0 = 26 \\
3 & 24 & -1 & -1 + \lfloor((24 + 1) \times 3) \div 2 ^ 2\rfloor = 17 & 24 + 1 = 25 \\
4 & 24 & 0 & 0 + \lfloor((24 - 0) \times 3) \div 2 ^ 2\rfloor = 18 & 24 - 0 = 24 \\
\end{tabular}
}
}

\clearpage

%This work is licensed under the
%Creative Commons Attribution-Share Alike 3.0 United States License.
%To view a copy of this license, visit
%http://creativecommons.org/licenses/by-sa/3.0/us/ or send a letter to
%Creative Commons,
%171 Second Street, Suite 300,
%San Francisco, California, 94105, USA.

\subsection{Computing QLP Coefficients and Residual}
\label{alac:compute_qlp_coeffs}
{\relsize{-2}
\ALGORITHM{a list of signed PCM samples, sample size, encoding parameters}{a list of 4 or 8 signed QLP coefficients, a block of residual data; or a \textit{residual overflow} exception}
\SetKwData{SAMPLES}{subframe samples}
\SetKwData{WINDOWED}{windowed}
\SetKwData{AUTOCORRELATION}{autocorrelated}
\SetKwData{LPCOEFF}{LP coefficient}
\SetKwData{QLPCOEFF}{QLP coefficient}
\SetKwData{SAMPLESIZE}{sample size}
\SetKwData{RESIDUAL}{residual}
\SetKwData{RESIDUALBLOCK}{residual block}
$\WINDOWED \leftarrow$ \hyperref[alac:window]{window signed integer \SAMPLES}\;
$\AUTOCORRELATION \leftarrow$ \hyperref[alac:autocorrelate]{autocorrelate \WINDOWED}\;
\eIf{$\text{\AUTOCORRELATION}_0 \neq 0.0$}{
  $\LPCOEFF \leftarrow$ \hyperref[alac:compute_lp_coeffs]{compute LP coefficients from \AUTOCORRELATION}\;
  $\text{\QLPCOEFF}_3 \leftarrow$ \hyperref[alac:quantize_lp_coeffs]{quantize $\text{\LPCOEFF}_3$ at order 4}\;
  $\text{\QLPCOEFF}_7 \leftarrow$ \hyperref[alac:quantize_lp_coeffs]{quantize $\text{\LPCOEFF}_7$ at order 8}\;
  $\text{\RESIDUAL}_3 \leftarrow$ \hyperref[alac:calculate_residuals]{calculate residuals from $\text{\QLPCOEFF}_3$ and \SAMPLES}\;
  $\text{\RESIDUAL}_7 \leftarrow$ \hyperref[alac:calculate_residuals]{calculate residuals from $\text{\QLPCOEFF}_7$ and \SAMPLES}\;
  $\text{\RESIDUALBLOCK}_3 \leftarrow$ \hyperref[alac:write_residuals]{encode residual block from $\text{\RESIDUAL}_3$ with \SAMPLESIZE}\;
  $\text{\RESIDUALBLOCK}_7 \leftarrow$ \hyperref[alac:write_residuals]{encode residual block from $\text{\RESIDUAL}_7$ with \SAMPLESIZE}\;
  \eIf{$\LEN(\text{\RESIDUALBLOCK}_3) < (\LEN(\text{\RESIDUALBLOCK}_7) + 64~bits)$}{
    \Return $\left\lbrace\begin{tabular}{l}
    $\text{\QLPCOEFF}_3$ \\
    $\text{\RESIDUALBLOCK}_3$ \\
    \end{tabular}\right.$\;
  }{
    \Return $\left\lbrace\begin{tabular}{l}
    $\text{\QLPCOEFF}_7$ \\
    $\text{\RESIDUALBLOCK}_7$ \\
    \end{tabular}\right.$\;
  }
}(\tcc*[f]{all samples are 0}){
  \QLPCOEFF $\leftarrow$ \texttt{[0, 0, 0, 0]}\;
  $\text{\RESIDUAL} \leftarrow$ \hyperref[alac:calculate_residuals]{calculate residuals from $\text{\QLPCOEFF}$ and \SAMPLES}\;
  $\text{\RESIDUALBLOCK} \leftarrow$ \hyperref[alac:write_residuals]{encode residual block from $\text{\RESIDUAL}$ with \SAMPLESIZE}\;
  \Return $\left\lbrace\begin{tabular}{l}
    $\text{\QLPCOEFF}$ \\
    $\text{\RESIDUALBLOCK}$ \\
  \end{tabular}\right.$\;
}
\EALGORITHM
}

\subsubsection{Windowing the Input Samples}
\label{alac:window}
{\relsize{-1}
\ALGORITHM{a list of signed input sample integers}{a list of signed windowed samples as floats}
\SetKwFunction{TUKEY}{tukey}
\SetKwData{SAMPLECOUNT}{sample count}
\SetKwData{WINDOWED}{windowed}
\SetKwData{SAMPLE}{sample}
\For{$i \leftarrow 0$ \emph{\KwTo}\SAMPLECOUNT}{
  $\text{\WINDOWED}_i = \text{\SAMPLE}_i \times \TUKEY(i)$\;
}
\Return \WINDOWED\;
\EALGORITHM
\par
\noindent
where the \VAR{Tukey} function is defined as:
\begin{equation*}
tukey(n) =
\begin{cases}
\frac{1}{2} \times \left[1 + cos\left(\pi \times \left(\frac{2 \times n}{\alpha \times (N - 1)} - 1 \right)\right)\right] & \text{ if } 0 \leq n \leq \frac{\alpha \times (N - 1)}{2} \\
1 & \text{ if } \frac{\alpha \times (N - 1)}{2} \leq n \leq (N - 1) \times (1 - \frac{\alpha}{2}) \\
\frac{1}{2} \times \left[1 + cos\left(\pi \times \left(\frac{2 \times n}{\alpha \times (N - 1)} - \frac{2}{\alpha} + 1 \right)\right)\right] & \text{ if } (N - 1) \times (1 - \frac{\alpha}{2}) \leq n \leq (N - 1) \\
\end{cases}
\end{equation*}
\par
\noindent
$N$ is the total number of input samples and $\alpha$ is $\nicefrac{1}{2}$.
\par
\noindent
{\relsize{-2}
\begin{tabular}{r|rcrcr}
$i$ & $\textsf{sample}_i$ & & \texttt{tukey}($i$) & & $\textsf{windowed}_i$ \\
\hline
0 & \texttt{0} & $\times$ & \texttt{0.00} & = & \texttt{0.00} \\
1 & \texttt{16} & $\times$ & \texttt{0.19} & = & \texttt{3.01} \\
2 & \texttt{31} & $\times$ & \texttt{0.61} & = & \texttt{18.95} \\
3 & \texttt{44} & $\times$ & \texttt{0.95} & = & \texttt{41.82} \\
4 & \texttt{54} & $\times$ & \texttt{1.00} & = & \texttt{54.00} \\
5 & \texttt{61} & $\times$ & \texttt{1.00} & = & \texttt{61.00} \\
6 & \texttt{64} & $\times$ & \texttt{1.00} & = & \texttt{64.00} \\
7 & \texttt{63} & $\times$ & \texttt{1.00} & = & \texttt{63.00} \\
8 & \texttt{58} & $\times$ & \texttt{1.00} & = & \texttt{58.00} \\
9 & \texttt{49} & $\times$ & \texttt{1.00} & = & \texttt{49.00} \\
10 & \texttt{38} & $\times$ & \texttt{1.00} & = & \texttt{38.00} \\
11 & \texttt{24} & $\times$ & \texttt{0.95} & = & \texttt{22.81} \\
12 & \texttt{8} & $\times$ & \texttt{0.61} & = & \texttt{4.89} \\
13 & \texttt{-8} & $\times$ & \texttt{0.19} & = & \texttt{-1.51} \\
14 & \texttt{-24} & $\times$ & \texttt{0.00} & = & \texttt{0.00} \\
\end{tabular}
}
}

\clearpage

\subsubsection{Autocorrelating Windowed Samples}
\label{alac:autocorrelate}
{\relsize{-1}
\ALGORITHM{a list of signed windowed samples}{a list of signed autocorrelation values}
\SetKwData{LAG}{lag}
\SetKwData{AUTOCORRELATION}{autocorrelated}
\SetKwData{TOTALSAMPLES}{total samples}
\SetKwData{WINDOWED}{windowed}
\For{$\LAG \leftarrow 0$ \emph{\KwTo}9}{
  $\text{\AUTOCORRELATION}_{\text{\LAG}} = \overset{\text{\TOTALSAMPLES} - \text{\LAG} - 1}{\underset{i = 0}{\sum}}\text{\WINDOWED}_i \times \text{\WINDOWED}_{i + \text{\LAG}}$\;
}
\Return \AUTOCORRELATION\;
\EALGORITHM
}

\subsubsection{Autocorrelation Example}
{\relsize{-1}
\begin{multicols}{2}
\begin{tabular}{rrrrr}
  \texttt{0.00} & $\times$ & \texttt{0.00} & $=$ & \texttt{0.00} \\
  \texttt{3.01} & $\times$ & \texttt{3.01} & $=$ & \texttt{9.07} \\
  \texttt{18.95} & $\times$ & \texttt{18.95} & $=$ & \texttt{359.07} \\
  \texttt{41.82} & $\times$ & \texttt{41.82} & $=$ & \texttt{1749.02} \\
  \texttt{54.00} & $\times$ & \texttt{54.00} & $=$ & \texttt{2916.00} \\
  \texttt{61.00} & $\times$ & \texttt{61.00} & $=$ & \texttt{3721.00} \\
  \texttt{64.00} & $\times$ & \texttt{64.00} & $=$ & \texttt{4096.00} \\
  \texttt{63.00} & $\times$ & \texttt{63.00} & $=$ & \texttt{3969.00} \\
  \texttt{58.00} & $\times$ & \texttt{58.00} & $=$ & \texttt{3364.00} \\
  \texttt{49.00} & $\times$ & \texttt{49.00} & $=$ & \texttt{2401.00} \\
  \texttt{38.00} & $\times$ & \texttt{38.00} & $=$ & \texttt{1444.00} \\
  \texttt{22.81} & $\times$ & \texttt{22.81} & $=$ & \texttt{520.37} \\
  \texttt{4.89} & $\times$ & \texttt{4.89} & $=$ & \texttt{23.91} \\
  \texttt{-1.51} & $\times$ & \texttt{-1.51} & $=$ & \texttt{2.27} \\
  \texttt{0.00} & $\times$ & \texttt{0.00} & $=$ & \texttt{0.00} \\
  \hline
  \multicolumn{3}{r}{$\textsf{autocorrelation}_0$} & $=$ & \texttt{24574.71} \\
\end{tabular}
\par
\begin{tabular}{rrrrr}
  \texttt{0.00} & $\times$ & \texttt{3.01} & $=$ & \texttt{0.00} \\
  \texttt{3.01} & $\times$ & \texttt{18.95} & $=$ & \texttt{57.08} \\
  \texttt{18.95} & $\times$ & \texttt{41.82} & $=$ & \texttt{792.48} \\
  \texttt{41.82} & $\times$ & \texttt{54.00} & $=$ & \texttt{2258.35} \\
  \texttt{54.00} & $\times$ & \texttt{61.00} & $=$ & \texttt{3294.00} \\
  \texttt{61.00} & $\times$ & \texttt{64.00} & $=$ & \texttt{3904.00} \\
  \texttt{64.00} & $\times$ & \texttt{63.00} & $=$ & \texttt{4032.00} \\
  \texttt{63.00} & $\times$ & \texttt{58.00} & $=$ & \texttt{3654.00} \\
  \texttt{58.00} & $\times$ & \texttt{49.00} & $=$ & \texttt{2842.00} \\
  \texttt{49.00} & $\times$ & \texttt{38.00} & $=$ & \texttt{1862.00} \\
  \texttt{38.00} & $\times$ & \texttt{22.81} & $=$ & \texttt{866.84} \\
  \texttt{22.81} & $\times$ & \texttt{4.89} & $=$ & \texttt{111.55} \\
  \texttt{4.89} & $\times$ & \texttt{-1.51} & $=$ & \texttt{-7.36} \\
  \texttt{-1.51} & $\times$ & \texttt{0.00} & $=$ & \texttt{0.00} \\
  \hline
  \multicolumn{3}{r}{$\textsf{autocorrelation}_1$} & $=$ & \texttt{23666.93} \\
\end{tabular}
\par
\begin{tabular}{rrrrr}
  \texttt{0.00} & $\times$ & \texttt{18.95} & $=$ & \texttt{0.00} \\
  \texttt{3.01} & $\times$ & \texttt{41.82} & $=$ & \texttt{125.97} \\
  \texttt{18.95} & $\times$ & \texttt{54.00} & $=$ & \texttt{1023.25} \\
  \texttt{41.82} & $\times$ & \texttt{61.00} & $=$ & \texttt{2551.10} \\
  \texttt{54.00} & $\times$ & \texttt{64.00} & $=$ & \texttt{3456.00} \\
  \texttt{61.00} & $\times$ & \texttt{63.00} & $=$ & \texttt{3843.00} \\
  \texttt{64.00} & $\times$ & \texttt{58.00} & $=$ & \texttt{3712.00} \\
  \texttt{63.00} & $\times$ & \texttt{49.00} & $=$ & \texttt{3087.00} \\
  \texttt{58.00} & $\times$ & \texttt{38.00} & $=$ & \texttt{2204.00} \\
  \texttt{49.00} & $\times$ & \texttt{22.81} & $=$ & \texttt{1117.77} \\
  \texttt{38.00} & $\times$ & \texttt{4.89} & $=$ & \texttt{185.82} \\
  \texttt{22.81} & $\times$ & \texttt{-1.51} & $=$ & \texttt{-34.36} \\
  \texttt{4.89} & $\times$ & \texttt{0.00} & $=$ & \texttt{0.00} \\
  \hline
  \multicolumn{3}{r}{$\textsf{autocorrelation}_2$} & $=$ & \texttt{21271.56} \\
\end{tabular}
\par
\begin{tabular}{rrrrr}
  \texttt{0.00} & $\times$ & \texttt{41.82} & $=$ & \texttt{0.00} \\
  \texttt{3.01} & $\times$ & \texttt{54.00} & $=$ & \texttt{162.65} \\
  \texttt{18.95} & $\times$ & \texttt{61.00} & $=$ & \texttt{1155.89} \\
  \texttt{41.82} & $\times$ & \texttt{64.00} & $=$ & \texttt{2676.56} \\
  \texttt{54.00} & $\times$ & \texttt{63.00} & $=$ & \texttt{3402.00} \\
  \texttt{61.00} & $\times$ & \texttt{58.00} & $=$ & \texttt{3538.00} \\
  \texttt{64.00} & $\times$ & \texttt{49.00} & $=$ & \texttt{3136.00} \\
  \texttt{63.00} & $\times$ & \texttt{38.00} & $=$ & \texttt{2394.00} \\
  \texttt{58.00} & $\times$ & \texttt{22.81} & $=$ & \texttt{1323.07} \\
  \texttt{49.00} & $\times$ & \texttt{4.89} & $=$ & \texttt{239.61} \\
  \texttt{38.00} & $\times$ & \texttt{-1.51} & $=$ & \texttt{-57.23} \\
  \texttt{22.81} & $\times$ & \texttt{0.00} & $=$ & \texttt{0.00} \\
  \hline
  \multicolumn{3}{r}{$\textsf{autocorrelation}_3$} & $=$ & \texttt{17970.57} \\
\end{tabular}
\par
\begin{tabular}{rrrrr}
  \texttt{0.00} & $\times$ & \texttt{54.00} & $=$ & \texttt{0.00} \\
  \texttt{3.01} & $\times$ & \texttt{61.00} & $=$ & \texttt{183.74} \\
  \texttt{18.95} & $\times$ & \texttt{64.00} & $=$ & \texttt{1212.74} \\
  \texttt{41.82} & $\times$ & \texttt{63.00} & $=$ & \texttt{2634.74} \\
  \texttt{54.00} & $\times$ & \texttt{58.00} & $=$ & \texttt{3132.00} \\
  \texttt{61.00} & $\times$ & \texttt{49.00} & $=$ & \texttt{2989.00} \\
  \texttt{64.00} & $\times$ & \texttt{38.00} & $=$ & \texttt{2432.00} \\
  \texttt{63.00} & $\times$ & \texttt{22.81} & $=$ & \texttt{1437.13} \\
  \texttt{58.00} & $\times$ & \texttt{4.89} & $=$ & \texttt{283.62} \\
  \texttt{49.00} & $\times$ & \texttt{-1.51} & $=$ & \texttt{-73.80} \\
  \texttt{38.00} & $\times$ & \texttt{0.00} & $=$ & \texttt{0.00} \\
  \hline
  \multicolumn{3}{r}{$\textsf{autocorrelation}_4$} & $=$ & \texttt{14231.18} \\
\end{tabular}
\end{multicols}
}

\clearpage

\subsubsection{LP Coefficient Calculation}
\label{alac:compute_lp_coeffs}
{\relsize{-1}
\ALGORITHM{a list of autocorrelation floats}{a list of LP coefficient lists}
\SetKwData{LPCOEFF}{LP coefficient}
\SetKwData{ERROR}{error}
\SetKwData{AUTOCORRELATION}{autocorrelation}
\begin{tabular}{rcl}
$\kappa_0$ &$\leftarrow$ & $ \AUTOCORRELATION_1 \div \AUTOCORRELATION_0$ \\
$\LPCOEFF_{0~0}$ &$\leftarrow$ & $ \kappa_0$ \\
$\ERROR_0$ &$\leftarrow$ & $ \AUTOCORRELATION_0 \times (1 - {\kappa_0} ^ 2)$ \\
\end{tabular}\;
\For{$i \leftarrow 1$ \emph{\KwTo}8}{
  \tcc{"zip" all of the previous row's LP coefficients
    \newline
    and the reversed autocorrelation values from 1 to i + 1
    \newline
    into ($c$,$a$) pairs
    \newline
    $q_i$ is $\AUTOCORRELATION_{i + 1}$ minus the sum of those multiplied ($c$,$a$) pairs}
  $q_i \leftarrow \AUTOCORRELATION_{i + 1}$\;
  \For{$j \leftarrow 0$ \emph{\KwTo}i}{
    $q_i \leftarrow q_i - (\LPCOEFF_{(i - 1)~j} \times \AUTOCORRELATION_{i - j})$\;
  }
  \BlankLine
  \tcc{"zip" all of the previous row's LP coefficients
    \newline
    and the previous row's LP coefficients reversed
    \newline
    into ($c$,$r$) pairs}
  $\kappa_i = q_i \div \ERROR_{i - 1}$\;
  \For{$j \leftarrow 0$ \emph{\KwTo}i}{
    \tcc{then build a new coefficient list of $c - (\kappa_i * r)$ for each ($c$,$r$) pair}
    $\LPCOEFF_{i~j} \leftarrow \LPCOEFF_{(i - 1)~j} - (\kappa_i \times \LPCOEFF_{(i - 1)~(i - j - 1)})$\;
  }
  $\text{\LPCOEFF}_{i~i} \leftarrow \kappa_i$\tcc*[r]{and append $\kappa_i$ as the final coefficient in that list}
  \BlankLine
  $\ERROR_i \leftarrow \ERROR_{i - 1} \times (1 - {\kappa_i}^2)$\;
}
\Return \LPCOEFF\;
\EALGORITHM
}

\begin{landscape}

\subsubsection{LP Coefficient Calculation Example}
\begin{table}[h]
{\relsize{-1}
\begin{tabular}{r|r}
$i$ & $\textsf{autocorrelation}_i$ \\
\hline
0 & \texttt{24598.25} \\
1 & \texttt{23694.34} \\
2 & \texttt{21304.57} \\
3 & \texttt{18007.86} \\
4 & \texttt{14270.30} \\
\end{tabular}
}
\end{table}

\begin{table}[h]
{\relsize{-1}
\renewcommand{\arraystretch}{1.45}
\begin{tabular}{|>{$}r<{$}||>{$}r<{$}|>{$}r<{$}|>{$}r<{$}|>{$}r<{$}|}
\hline
k_0 &
\multicolumn{4}{>{$}l<{$}|}{\texttt{23694.34} \div \texttt{24598.25} = \texttt{0.96}} \\
\textsf{LP coefficient}_{0~0} & \texttt{\color{blue}0.96} & & & \\
\textsf{error}_0 &
\multicolumn{4}{>{$}l<{$}|}{\texttt{24598.25} \times (1 - \texttt{0.96} ^ 2) = \texttt{1774.62}} \\
\hline
q_1 & \multicolumn{4}{>{$}l<{$}|}{\texttt{21304.57} - (\texttt{0.96} \times \texttt{23694.34}) = \texttt{-1519.07}} \\
k_1 & \multicolumn{4}{>{$}l<{$}|}{\texttt{-1519.07} \div \texttt{1774.62} = \texttt{-0.86}} \\
\textsf{LP coefficient}_{1~i} &
\texttt{0.96} -(\texttt{-0.86} \times \texttt{0.96}) = \texttt{\color{blue}1.79} &
\texttt{\color{blue}-0.86} & & \\
\textsf{error}_1 & \multicolumn{4}{>{$}l<{$}|}{\texttt{1774.62} \times (1 - \texttt{-0.86} ^ 2) = \texttt{474.30}} \\
\hline
q_2 & \multicolumn{4}{>{$}l<{$}|}{\texttt{18007.86} - (\texttt{1.79} \times \texttt{21304.57} + \texttt{-0.86} \times \texttt{23694.34}) = \texttt{201.96}} \\
k_2 & \multicolumn{4}{>{$}l<{$}|}{\texttt{201.96} \div \texttt{474.30} = \texttt{0.43}} \\
\textsf{LP coefficient}_{2~i} &
\texttt{1.79} -(\texttt{0.43} \times \texttt{-0.86}) = \texttt{\color{blue}2.15} &
\texttt{-0.86} -(\texttt{0.43} \times \texttt{1.79}) = \texttt{\color{blue}-1.62} &
\texttt{\color{blue}0.43} & \\
\textsf{error}_2 & \multicolumn{4}{>{$}l<{$}|}{\texttt{474.30} \times (1 - \texttt{0.43} ^ 2) = \texttt{388.31}} \\
\hline
q_3 & \multicolumn{4}{>{$}l<{$}|}{\texttt{14270.30} - (\texttt{2.15} \times \texttt{18007.86} + \texttt{-1.62} \times \texttt{21304.57} + \texttt{0.43} \times \texttt{23694.34}) = \texttt{-122.06}} \\
k_3 & \multicolumn{4}{>{$}l<{$}|}{\texttt{-122.06} \div \texttt{388.31} = \texttt{-0.31}} \\
\textsf{LP coefficient}_{3~i} &
\texttt{2.15} -(\texttt{-0.31} \times \texttt{0.43}) = \texttt{\color{blue}2.29} &
\texttt{-1.62} -(\texttt{-0.31} \times \texttt{-1.62}) = \texttt{\color{blue}-2.13} &
\texttt{0.43} -(\texttt{-0.31} \times \texttt{2.15}) = \texttt{\color{blue}1.10} &
\texttt{\color{blue}-0.31} \\
\textsf{error}_3 & \multicolumn{4}{>{$}l<{$}|}{\texttt{388.31} \times (1 - \texttt{-0.31} ^ 2) = \texttt{349.94}} \\
\hline
\end{tabular}
\renewcommand{\arraystretch}{1.0}
}
\end{table}

\end{landscape}

\subsubsection{Quantizing LP Coefficients}
\label{alac:quantize_lp_coeffs}
\ALGORITHM{LP coefficients, an order value of 4 or 8}{QLP coefficients as a list of signed integers}
\SetKwData{ORDER}{order}
\SetKwFunction{MIN}{min}
\SetKwFunction{MAX}{max}
\SetKwFunction{ROUND}{round}
\SetKwData{QLPMIN}{QLP min}
\SetKwData{QLPMAX}{QLP max}
\SetKwData{LPCOEFF}{LP coefficient}
\SetKwData{QLPCOEFF}{QLP coefficient}
\tcc{QLP min and max are the smallest and largest QLP coefficients that fit in a signed field that's 16 bits wide}
$\QLPMIN \leftarrow 2 ^ \text{15} - 1$\;
$\QLPMAX \leftarrow -(2 ^ \text{15})$\;
$e \leftarrow 0.0$\;
\For{$i \leftarrow 0$ \emph{\KwTo}\ORDER}{
  $e \leftarrow e + \text{\LPCOEFF}_{\ORDER - 1~i} \times 2 ^ 9$\;
  $\text{\QLPCOEFF}_i \leftarrow \MIN(\MAX(\ROUND(e)~,~\text{\QLPMIN})~,~\text{\QLPMAX})$\;
  $e \leftarrow e - \text{\QLPCOEFF}_i$\;
}
\Return \QLPCOEFF\;
\EALGORITHM

\clearpage

\subsubsection{Quantizing Coefficients Example}
\begin{align*}
e &\leftarrow \texttt{0.00} + \texttt{2.29} \times 2 ^ 9 = \texttt{1170.49} \\
\textsf{QLP coefficient}_0 &\leftarrow \texttt{round}(\texttt{1170.49}) = \texttt{\color{blue}1170} \\
e &\leftarrow \texttt{1170.49} - 1170 = \texttt{0.49} \\
e &\leftarrow \texttt{0.49} + \texttt{-2.13} \times 2 ^ 9 = \texttt{-1087.81} \\
\textsf{QLP coefficient}_1 &\leftarrow \texttt{round}(\texttt{-1087.81}) = \texttt{\color{blue}-1088} \\
e &\leftarrow \texttt{-1087.81} - -1088 = \texttt{0.19} \\
e &\leftarrow \texttt{0.19} + \texttt{1.10} \times 2 ^ 9 = \texttt{564.59} \\
\textsf{QLP coefficient}_2 &\leftarrow\texttt{round}(\texttt{564.59}) = \texttt{\color{blue}565} \\
e &\leftarrow \texttt{564.59} - 565 = \texttt{-0.41} \\
e &\leftarrow \texttt{-0.41} + \texttt{-0.31} \times 2 ^ 9 = \texttt{-161.35} \\
\textsf{QLP coefficient}_3 &\leftarrow \texttt{round}(\texttt{-161.35}) = \texttt{\color{blue}-161} \\
e &\leftarrow \texttt{-161.35} - -161 = \texttt{-0.35} \\
\end{align*}


\clearpage

%This work is licensed under the
%Creative Commons Attribution-Share Alike 3.0 United States License.
%To view a copy of this license, visit
%http://creativecommons.org/licenses/by-sa/3.0/us/ or send a letter to
%Creative Commons,
%171 Second Street, Suite 300,
%San Francisco, California, 94105, USA.

\subsection{Residual Encoding}
\label{flac:write_residual_block}
{\relsize{-1}
\ALGORITHM{a set of signed residual values, the subframe's block size and predictor order, minimum and maximum partition order from encoding parameters}{an encoded block of residuals}
\SetKwData{MINPORDER}{minimum partition order}
\SetKwData{MAXPORDER}{maximum partition order}
\SetKwData{ORDER}{predictor order}
\SetKwData{BLOCKSIZE}{block size}
\SetKwData{PAORDER}{partition order}
\SetKwData{PSIZE}{partitions size}
\SetKwData{PSUM}{partition sum}
\SetKwData{RICE}{Rice}
\SetKwData{CODING}{coding method}
\SetKwData{UNSIGNED}{unsigned}
\SetKwData{PARTITION}{partition}
\SetKwData{PLEN}{partition length}
\SetKwData{MSB}{MSB}
\SetKwData{LSB}{LSB}
\SetKwFunction{SUM}{sum}
\SetKwFunction{MAX}{max}
\SetKw{BREAK}{break}
\tcc{generate set of partitions for each partition order}
\For{$o \leftarrow \text{\MINPORDER}$ \emph{\KwTo}(\MAXPORDER + 1)}{
  \eIf{$(\BLOCKSIZE \bmod 2^{o}) = 0$}{
    $\left.\begin{tabular}{r}
      $\text{\RICE}_o$ \\
      $\text{\PARTITION}_o$ \\
      $\text{\PSIZE}_o$ \\
    \end{tabular}\right\rbrace \leftarrow$ \hyperref[flac:write_residual_partition]{encode residual partitions from}
    $\left\lbrace\begin{tabular}{l}
    partition order $o$ \\
    \textsf{predictor order} \\
    \textsf{residual values} \\
    \BLOCKSIZE \\
    \end{tabular}\right.$
  }{
    \BREAK\;
  }
}
\BlankLine
choose partition order $o$ such that $\PSIZE_{o}$ is smallest\;
\BlankLine
\eIf{$\MAX(\text{\RICE}_{o}) > 14$}{
  $\CODING \leftarrow 1$\;
}{
  $\CODING \leftarrow 0$\;
}
\BlankLine
\tcc{write 1 or more residual partitions to residual block}
$\CODING \rightarrow$ \WRITE 2 unsigned bits\;
$o \rightarrow$ \WRITE 4 unsigned bits\;
\For{$p \leftarrow 0$ \emph{\KwTo}$2 ^ {o}$} {
  \eIf{$\CODING = 0$}{
    $\text{\RICE}_{o~p} \rightarrow$ \WRITE 4 unsigned bits\;
  }{
    $\text{\RICE}_{o~p} \rightarrow$ \WRITE 5 unsigned bits\;
  }
  \BlankLine
  \eIf{$p = 0$}{
    $\text{\PLEN}_{o~0} \leftarrow \BLOCKSIZE \div 2 ^ {o} - \ORDER$\;
  }{
    $\text{\PLEN}_{o~p} \leftarrow \BLOCKSIZE \div 2 ^ {o}$\;
  }
  \BlankLine
  \For(\tcc*[f]{write residual partition}){$i \leftarrow 0$ \emph{\KwTo}$\text{\PLEN}_{o~p}$}{
    \eIf{$\text{\PARTITION}_{o~p~i} \geq 0$}{
      $\text{\UNSIGNED}_i \leftarrow \text{\PARTITION}_{o~p~i} \times 2$\;
    }{
    $\text{\UNSIGNED}_i \leftarrow (-\text{\PARTITION}_{o~p~i} - 1) \times 2 + 1$\;
    }
    $\text{\MSB}_i \leftarrow \lfloor \text{\UNSIGNED}_i \div 2 ^ \text{\RICE} \rfloor$\;
    $\text{\LSB}_i \leftarrow \text{\UNSIGNED}_i - (\text{\MSB}_i \times 2 ^ \text{\RICE})$\;
    $\text{\MSB}_i \rightarrow$ \WUNARY with stop bit 1\;
    $\text{\LSB}_i \rightarrow$ \WRITE $\text{\RICE}$ unsigned bits\;
  }
}
\Return encoded residual block\;
\EALGORITHM
}

\clearpage

\subsubsection{Encoding Partitions}
\label{flac:write_residual_partition}
{\relsize{-1}
\ALGORITHM{partition order $o$, predictor order, residual values, block size, maximum Rice parameter from encoding parameters}{Rice parameter, 1 or more residual partitions, total estimated size}
\SetKwData{ORDER}{predictor order}
\SetKwData{BLOCKSIZE}{block size}
\SetKwData{PSIZE}{partitions size}
\SetKwData{PLEN}{plength}
\SetKwData{PARTITION}{partition}
\SetKwData{RESIDUAL}{residual}
\SetKwData{PSUM}{partition sum}
\SetKwData{RICE}{Rice}
\SetKwData{MAXPARAMETER}{maximum Rice parameter}
\SetKw{BREAK}{break}
$\text{\PSIZE} \leftarrow 0$\;
\BlankLine
\For(\tcc*[f]{split residuals into partitions}){$p \leftarrow 0$ \emph{\KwTo}$2 ^ {o}$}{
  \eIf{$p = 0$}{
    $\text{\PLEN}_{0} \leftarrow \BLOCKSIZE \div 2 ^ {o} - \ORDER$\;
  }{
    $\text{\PLEN}_{p} \leftarrow \BLOCKSIZE \div 2 ^ {o}$\;
  }
  $\text{\PARTITION}_{p} \leftarrow$ get next $\text{\PLEN}_{p}$ values from \RESIDUAL\;
  \BlankLine
  $\text{\PSUM}_{p} \leftarrow \overset{\text{\PLEN}_{p} - 1}{\underset{i = 0}{\sum}} |\text{\PARTITION}_{p~i}|$\;
  \BlankLine
  $\text{\RICE}_{p} \leftarrow 0$\tcc*[r]{compute best Rice parameter for partition}
  \While{$\text{\PLEN}_{p} \times 2 ^ {\text{\RICE}_{p}} < \text{\PSUM}_{p}$}{
    \eIf{$\text{\RICE}_{p} < \MAXPARAMETER$}{
      $\text{\RICE}_{p} \leftarrow \text{\RICE}_{p} + 1$\;
    }{
      \BREAK\;
    }
  }
  \BlankLine
  \eIf(\tcc*[f]{add estimated size of partition to total size}){$\text{\RICE}_{p} > 0$}{
    $\text{\PSIZE} \leftarrow \text{\PSIZE} + 4 + ((1 + \text{\RICE}_{p}) \times \text{\PLEN}_{p}) + \left\lfloor\frac{\text{\PSUM}_{p}}{2 ^ {\text{\RICE}_{p} - 1}}\right\rfloor - \left\lfloor\frac{\text{\PLEN}_{p}}{2}\right\rfloor$\;
  }{
    $\text{\PSIZE} \leftarrow \text{\PSIZE} + 4 + \text{\PLEN}_{p} + (\text{\PSUM}_{p} \times 2) - \left\lfloor\frac{\text{\PLEN}_{p}}{2}\right\rfloor$\;
  }
}
\BlankLine
\Return $\left\lbrace\begin{tabular}{l}
$\text{\RICE}$ \\
$\text{\PARTITION}$ \\
$\text{\PSIZE}$ \\
\end{tabular}\right.$\;
\EALGORITHM
}

\begin{figure}[h]
\includegraphics{flac/figures/residual.pdf}
\end{figure}

\clearpage

\subsubsection{Residual Encoding Example}
Given a set of residuals \texttt{[2, 6, -2, 0, -1, -2, 3, -1, -3]},
block size of 10 and predictor order of 1:
{\relsize{-1}
  \begin{align*}
  \intertext{$\text{partition order}~o = 0$:}
  \textsf{plength}_{0~0} &\leftarrow 10 \div 2 ^ 0 - 1 = 9 \\
  \textsf{partition}_{0~0} &\leftarrow \texttt{[2, 6, -2, 0, -1, -2, 3, -1, -3]} \\
  \textsf{partition sum}_{0~0} &\leftarrow 2 + 6 + 2 + 0 + 1 + 2 + 3 + 1 + 3 = 20 \\
  \textsf{Rice}_{0~0} &\leftarrow \textbf{1}~~(9 \times 2 ^ \textbf{1} < 20 \text{ and } 9 \times 2 ^ \textbf{2} > 20) \\
  \textsf{partitions size}_0 &\leftarrow 0 + 4 + ((1 + 1) \times 9) + \left\lfloor\frac{20}{2 ^ 1 - 1}\right\rfloor - \left\lfloor\frac{9}{2}\right\rfloor = \textbf{38} \\
  \intertext{$\text{partition order}~o = 1$:}
  \textsf{plength}_{1~0} &\leftarrow 10 \div 2 ^ 1 - 1 = 4 \\
  \textsf{partition}_{1~0} &\leftarrow \texttt{[2, 6, -2, 0]} \\
  \textsf{partition sum}_{1~0} &\leftarrow 2 + 6 + 2 + 0 = 10 \\
  \textsf{Rice}_{1~0} &\leftarrow \textbf{1}~~(4 \times 2 ^ \textbf{1} < 10 \text{ and } 4 \times 2 ^ \textbf{2} > 10) \\
  \textsf{partitions size}_1 &\leftarrow 0 + 4 + ((1 + 1) \times 4) + \left\lfloor\frac{10}{2 ^ 1 - 1}\right\rfloor - \left\lfloor\frac{4}{2}\right\rfloor = \textbf{20} \\
  \textsf{plength}_{1~1} &\leftarrow 10 \div 2 ^ 1 = 5 \\
  \textsf{partition}_{1~1} &\leftarrow \texttt{[-1, -2, 3, -1, -3]} \\
  \textsf{partition sum}_{1~1} &\leftarrow 1 + 2 + 3 + 1 + 3 = 10 \\
  \textsf{Rice}_{1~1} &\leftarrow \textbf{0}~~(5 \times 2 ^ \textbf{0} < 10 \text{ and } 5 \times 2 ^ \textbf{1} = 10) \\
  \textsf{partitions size}_1 &\leftarrow \textbf{20} + 4 + 5 + (10 \times
  2) - \left\lfloor\frac{5}{2}\right\rfloor = \textbf{47}
\end{align*}}
\par
\noindent
Since block size of $10 \bmod 2 ^ 2 \neq 0$, we stop at partition order 1
because the list of residuals can't be divided equally into more partitions.
And because $\textsf{partitions size}_0$ of 38 is smaller than
$\textsf{partitions size}_1$ of 47, we use partition order 0
to encode our residuals into a single partition with 9 residuals.

\begin{figure}[h]
  \includegraphics{flac/figures/residuals-enc-example.pdf}
\end{figure}


\subsection{Writing Subframe Header}
\label{alac:write_subframe_header}
\ALGORITHM{4 or 8 signed QLP coefficients}{a subframe header}
\SetKwData{COEFFCOUNT}{coefficient count}
\SetKwData{COEFF}{QLP coefficient}
$0 \rightarrow$ \WRITE 4 unsigned bits\tcc*[r]{prediction type}
$9 \rightarrow$ \WRITE 4 unsigned bits\tcc*[r]{QLP shift needed}
$4 \rightarrow$ \WRITE 3 unsigned bits\tcc*[r]{Rice modifier}
$\text{\COEFFCOUNT} \rightarrow$ \WRITE 5 unsigned bits\;
\For{$i \leftarrow 0$ \emph{\KwTo}\COEFFCOUNT}{
  $\text{\COEFF}_i \rightarrow$ \WRITE 16 signed bits\;
}
\Return subframe header data\;
\EALGORITHM
\begin{figure}[h]
\includegraphics{alac/figures/subframe_header.pdf}
\end{figure}
\par
\noindent
For example, given the QLP coefficients
\texttt{1170, -1088, 565, -161},
the subframe header is written as:
\begin{figure}[h]
\includegraphics{alac/figures/subframe-build.pdf}
\end{figure}
