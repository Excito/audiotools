%This work is licensed under the
%Creative Commons Attribution-Share Alike 3.0 United States License.
%To view a copy of this license, visit
%http://creativecommons.org/licenses/by-sa/3.0/us/ or send a letter to
%Creative Commons,
%171 Second Street, Suite 300,
%San Francisco, California, 94105, USA.

\section{Metadata Blocks}
\begin{figure}[h]
  \includegraphics{flac/figures/metadata-blocks.pdf}
\end{figure}
\par
\noindent
\VAR{last block} of \texttt{1} indicates the block is the last one
in the set of metadata blocks.
\VAR{block size} indicates the size of the metadata block data,
not including its 32 bit header.
\VAR{block type} is one of the following:
\begin{table}[h]
\begin{tabular}{r | l}
  block type & block \\
  \hline
  \texttt{0} & STREAMINFO \\
  \texttt{1} & PADDING \\
  \texttt{2} & APPLICATION \\
  \texttt{3} & SEEKTABLE \\
  \texttt{4} & VORBIS\_COMMENT \\
  \texttt{5} & CUESHEET \\
  \texttt{6} & PICTURE \\
  \texttt{7-126} & reserved \\
  \texttt{127} & invalid \\
\end{tabular}
\end{table}

\clearpage

\subsection{STREAMINFO}
\begin{figure}[h]
\includegraphics{flac/figures/streaminfo.pdf}
\end{figure}
\par
\noindent
STREAMINFO must always be the first metadata block and always
has a block size of 34 bytes.
\VAR{minimum block size} and \VAR{maximum block size}
indicate the length of the largest and smallest
frame in the stream, in PCM frames.
For streams with a fixed block size (which are the most common),
these numbers will be the same.
\VAR{minimum frame size} and \VAR{maximum frame size}
indicate the size of the largest and smallest frame in the stream,
in bytes.
\VAR{sample rate}, \VAR{channels} and \VAR{bits per sample}
indicate the stream's characteristics.
Finally, \VAR{MD5 sum} is a hash of the stream's raw decoded data
when its PCM sample integers are decoded to signed, little-endian bytes.

\clearpage

\subsubsection{STREAMINFO example}
\begin{figure}[h]
  \includegraphics{flac/figures/streaminfo-example.pdf}
\end{figure}
\begin{tabular}{rrl}
last block & \texttt{0} & not last block in set of blocks \\
block type & \texttt{0} & STREAMINFO \\
block size & \texttt{34} & bytes \\
minimum block size & \texttt{4096} & PCM frames \\
maximum block size & \texttt{4096} & PCM frames \\
minimum frame size & \texttt{1542} & bytes \\
maximum frame size & \texttt{8546} & bytes \\
sample rate & \texttt{44100} & Hz \\
channels & \texttt{1} & $+ 1 = 2$ \\
bits per sample & \texttt{15} & $+ 1 = 16$ \\
total PCM frames & \texttt{304844} & \\
MD5 sum & \multicolumn{2}{l}{\texttt{FAF2692FFDEC2D5B300176B462887D92}} \\
\end{tabular}

\clearpage

\subsection{PADDING}
\begin{figure}[h]
  \includegraphics{flac/figures/padding.pdf}
\end{figure}
\par
\noindent
PADDING is simply a block full of NULL (\texttt{0x00}) bytes.
Its purpose is to provide extra metadata space within the FLAC file.
By having a padding block, other metadata blocks can be grown or
shrunk without having to rewrite the entire FLAC file by removing or
adding space to the padding.

\subsection{APPLICATION}
\begin{figure}[h]
\includegraphics{flac/figures/application.pdf}
\end{figure}
\noindent
APPLICATION is a general-purpose metadata block used by a variety of
different programs.
Its contents are defined by the 4 byte ASCII Application ID value.
\begin{table}[h]
{\relsize{-1}
\begin{tabular}{rl}
\texttt{Fica} & CUE Splitter \\
\texttt{Ftol} & flac-tools \\
\texttt{aiff} & FLAC AIFF chunk storage \\
\texttt{imag} & flac-image application for storing arbitrary files in APPLICATION metadata blocks \\
\texttt{riff} & FLAC RIFF chunk storage \\
\texttt{xmcd} & xmcd \\
\end{tabular}
}
\caption{some defined application IDs}
\end{table}

\clearpage

\subsection{SEEKTABLE}
\begin{figure}[h]
\includegraphics{flac/figures/seektable.pdf}
\end{figure}
\par
\noindent
There are $\textsf{block size} \div 18$ seek points in the SEEKTABLE block.
Each seek point indicates a position in the FLAC stream.
\VAR{sample number} indicates the seek point's sample number, in PCM frames.
\VAR{byte offset} indicates the seek point's
byte position in the stream starting from the beginning
of the first FLAC frame (\textit{not} the beginning of the FLAC file itself).
\VAR{samples} indicates the number of PCM frames
in the seek point's FLAC frame.
\subsubsection{SEEKTABLE example}
\begin{figure}[h]
  \includegraphics{flac/figures/seektable-example.pdf}
\end{figure}
\begin{table}[h]
{\relsize{-1}
\begin{tabular}{r|rrr}
$i$ & $\text{sample number}_i$ & $\text{byte offset}_i$ & $\text{samples}_i$ \\
\hline
0 & 0 & 0 & 4096 \\
1 & 438272 & 436207 & 4096 \\
2 & 880640 & 879604 & 4096 \\
3 & 1318912 & 1318778 & 4096 \\
4 & 1761280 & 1763423 & 4096 \\
5 & 2203648 & 2205395 & 4096 \\
\end{tabular}
}
\end{table}

\clearpage

\subsection{VORBIS\_COMMENT}
\begin{figure}[h]
\includegraphics{flac/figures/vorbiscomment.pdf}
\end{figure}
\par
\noindent
The blue fields are all stored little-endian.
Comment strings are \texttt{"KEY=value"} pairs
where \texttt{KEY} is an ASCII value in the range \texttt{0x20}
through \texttt{0x7D}, excluding \texttt{0x3D},
is case-insensitive and may occur in multiple comment strings.
\texttt{value} is a UTF-8 value.
\begin{table}[h]
  {\relsize{-1}
    \begin{tabular}{rl}
      \texttt{ALBUM} & album name \\
      \texttt{ARTIST} & artist name \\
      \texttt{CATALOG} & CD spine number \\
      \texttt{COMPOSER} & the work's author \\
      \texttt{COMMENT} & a short comment \\
      \texttt{CONDUCTOR} & performing ensemble's leader \\
      \texttt{COPYRIGHT} & copyright attribution \\
      \texttt{DATE} & recording date \\
      \texttt{DISCNUMBER} & disc number for multi-volume work \\
      \texttt{DISCTOTAL} & disc total for multi-volume work \\
      \texttt{GENRE} & a short music genre label \\
      \texttt{ISRC} & ISRC number for the track \\
      \texttt{PERFORMER} & performer name, orchestra, actor, etc. \\
      \texttt{PUBLISHER} & album's publisher \\
      \texttt{SOURCE MEDIUM} & CD, radio, cassette, vinyl LP, etc. \\
      \texttt{TITLE} & track name \\
      \texttt{TRACKNUMBER} & track number \\
      \texttt{TRACKTOTAL} & total number of tracks \\
    \end{tabular}
  }
\end{table}

\clearpage

\subsubsection{VORBIS\_COMMENT Example}
\begin{figure}[h]
  \includegraphics{flac/figures/vorbiscomment-example.pdf}
\end{figure}
\begin{table}[h]
\begin{tabular}{rl}
  vendor string : & \texttt{reference libFLAC 1.2.1 20070917} \\
  $\text{comment}_0$ : & \texttt{ALBUM=album} \\
  $\text{comment}_1$ : & \texttt{TRACKNUMBER=1} \\
  $\text{comment}_2$ : & \texttt{TITLE=track name} \\
  $\text{comment}_3$ : & \texttt{ARTIST=artist name} \\
\end{tabular}
\end{table}

\clearpage

%% \subsubsection{CUESHEET Example}

%% \begin{figure}[h]
%%   \includegraphics{flac/figures/cuesheet-example.pdf}
%% \end{figure}

%% \clearpage

\subsection{PICTURE}
\begin{figure}[h]
\includegraphics{flac/figures/picture.pdf}
\end{figure}
{\relsize{-1}
\begin{tabular}{r|l}
picture type & type \\
\hline
0 & Other \\
1 & 32x32 pixels `file icon' (PNG only) \\
2 & Other file icon \\
3 & Cover (front) \\
4 & Cover (back) \\
5 & Leaflet page \\
6 & Media (e.g. label side of CD) \\
7 & Lead artist / Lead performer / Soloist \\
8 & Artist / Performer \\
9 & Conductor \\
10 & Band / Orchestra \\
11 & Composer \\
12 & Lyricist / Text writer \\
13 & Recording location \\
14 & During recording \\
15 & During performance \\
16 & Movie / Video screen capture \\
17 & A bright colored fish \\
18 & Illustration \\
19 & Band / Artist logotype \\
20 & Publisher / Studio logotype \\
\end{tabular}
}
\clearpage

\subsubsection{PICTURE Example}
\begin{figure}[h]
  \includegraphics{flac/figures/picture-example.pdf}
\end{figure}
\begin{table}[h]
{\relsize{-1}
  \begin{tabular}{rl}
    MIME type : & \texttt{"image/png"} \\
    description : & \texttt{""} (empty string) \\
    width : & \texttt{11} pixels \\
    height : & \texttt{12} pixels \\
    color depth : & \texttt{8} bits per channel \\
    color count : & \texttt{0} (not an indexed color image) \\
    data length : & \texttt{125} bytes \\
  \end{tabular}
}
\end{table}

\clearpage

\subsection{CUESHEET}
\begin{figure}[h]
  \includegraphics{flac/figures/cuesheet.pdf}
\end{figure}
