%This work is licensed under the
%Creative Commons Attribution-Share Alike 3.0 United States License.
%To view a copy of this license, visit
%http://creativecommons.org/licenses/by-sa/3.0/us/ or send a letter to
%Creative Commons,
%171 Second Street, Suite 300,
%San Francisco, California, 94105, USA.

\subsection{Residual Encoding}
\label{flac:write_residual_block}
{\relsize{-1}
\ALGORITHM{a set of signed residual values, the subframe's block size and predictor order, minimum and maximum partition order from encoding parameters}{an encoded block of residuals}
\SetKwData{MINPORDER}{minimum partition order}
\SetKwData{MAXPORDER}{maximum partition order}
\SetKwData{ORDER}{predictor order}
\SetKwData{BLOCKSIZE}{block size}
\SetKwData{PAORDER}{partition order}
\SetKwData{PSIZE}{partitions size}
\SetKwData{PSUM}{partition sum}
\SetKwData{RICE}{Rice}
\SetKwData{CODING}{coding method}
\SetKwData{UNSIGNED}{unsigned}
\SetKwData{PARTITION}{partition}
\SetKwData{PLEN}{partition length}
\SetKwData{MSB}{MSB}
\SetKwData{LSB}{LSB}
\SetKwFunction{SUM}{sum}
\SetKwFunction{MAX}{max}
\SetKw{BREAK}{break}
\tcc{generate set of partitions for each partition order}
\For{$o \leftarrow \text{\MINPORDER}$ \emph{\KwTo}(\MAXPORDER + 1)}{
  \eIf{$(\BLOCKSIZE \bmod 2^{o}) = 0$}{
    $\left.\begin{tabular}{r}
      $\text{\RICE}_o$ \\
      $\text{\PARTITION}_o$ \\
      $\text{\PSIZE}_o$ \\
    \end{tabular}\right\rbrace \leftarrow$ \hyperref[flac:write_residual_partition]{encode residual partitions from}
    $\left\lbrace\begin{tabular}{l}
    partition order $o$ \\
    \textsf{predictor order} \\
    \textsf{residual values} \\
    \BLOCKSIZE \\
    \end{tabular}\right.$
  }{
    \BREAK\;
  }
}
\BlankLine
choose partition order $o$ such that $\PSIZE_{o}$ is smallest\;
\BlankLine
\eIf{$\MAX(\text{\RICE}_{o}) > 14$}{
  $\CODING \leftarrow 1$\;
}{
  $\CODING \leftarrow 0$\;
}
\BlankLine
\tcc{write 1 or more residual partitions to residual block}
$\CODING \rightarrow$ \WRITE 2 unsigned bits\;
$o \rightarrow$ \WRITE 4 unsigned bits\;
\For{$p \leftarrow 0$ \emph{\KwTo}$2 ^ {o}$} {
  \eIf{$\CODING = 0$}{
    $\text{\RICE}_{o~p} \rightarrow$ \WRITE 4 unsigned bits\;
  }{
    $\text{\RICE}_{o~p} \rightarrow$ \WRITE 5 unsigned bits\;
  }
  \BlankLine
  \eIf{$p = 0$}{
    $\text{\PLEN}_{o~0} \leftarrow \BLOCKSIZE \div 2 ^ {o} - \ORDER$\;
  }{
    $\text{\PLEN}_{o~p} \leftarrow \BLOCKSIZE \div 2 ^ {o}$\;
  }
  \BlankLine
  \For(\tcc*[f]{write residual partition}){$i \leftarrow 0$ \emph{\KwTo}$\text{\PLEN}_{o~p}$}{
    \eIf{$\text{\PARTITION}_{o~p~i} \geq 0$}{
      $\text{\UNSIGNED}_i \leftarrow \text{\PARTITION}_{o~p~i} \times 2$\;
    }{
    $\text{\UNSIGNED}_i \leftarrow (-\text{\PARTITION}_{o~p~i} - 1) \times 2 + 1$\;
    }
    $\text{\MSB}_i \leftarrow \lfloor \text{\UNSIGNED}_i \div 2 ^ \text{\RICE} \rfloor$\;
    $\text{\LSB}_i \leftarrow \text{\UNSIGNED}_i - (\text{\MSB}_i \times 2 ^ \text{\RICE})$\;
    $\text{\MSB}_i \rightarrow$ \WUNARY with stop bit 1\;
    $\text{\LSB}_i \rightarrow$ \WRITE $\text{\RICE}$ unsigned bits\;
  }
}
\Return encoded residual block\;
\EALGORITHM
}

\clearpage

\subsubsection{Encoding Partitions}
\label{flac:write_residual_partition}
{\relsize{-1}
\ALGORITHM{partition order $o$, predictor order, residual values, block size, maximum Rice parameter from encoding parameters}{Rice parameter, 1 or more residual partitions, total estimated size}
\SetKwData{ORDER}{predictor order}
\SetKwData{BLOCKSIZE}{block size}
\SetKwData{PSIZE}{partitions size}
\SetKwData{PLEN}{plength}
\SetKwData{PARTITION}{partition}
\SetKwData{RESIDUAL}{residual}
\SetKwData{PSUM}{partition sum}
\SetKwData{RICE}{Rice}
\SetKwData{MAXPARAMETER}{maximum Rice parameter}
\SetKw{BREAK}{break}
$\text{\PSIZE} \leftarrow 0$\;
\BlankLine
\For(\tcc*[f]{split residuals into partitions}){$p \leftarrow 0$ \emph{\KwTo}$2 ^ {o}$}{
  \eIf{$p = 0$}{
    $\text{\PLEN}_{0} \leftarrow \BLOCKSIZE \div 2 ^ {o} - \ORDER$\;
  }{
    $\text{\PLEN}_{p} \leftarrow \BLOCKSIZE \div 2 ^ {o}$\;
  }
  $\text{\PARTITION}_{p} \leftarrow$ get next $\text{\PLEN}_{p}$ values from \RESIDUAL\;
  \BlankLine
  $\text{\PSUM}_{p} \leftarrow \overset{\text{\PLEN}_{p} - 1}{\underset{i = 0}{\sum}} |\text{\PARTITION}_{p~i}|$\;
  \BlankLine
  $\text{\RICE}_{p} \leftarrow 0$\tcc*[r]{compute best Rice parameter for partition}
  \While{$\text{\PLEN}_{p} \times 2 ^ {\text{\RICE}_{p}} < \text{\PSUM}_{p}$}{
    \eIf{$\text{\RICE}_{p} < \MAXPARAMETER$}{
      $\text{\RICE}_{p} \leftarrow \text{\RICE}_{p} + 1$\;
    }{
      \BREAK\;
    }
  }
  \BlankLine
  \eIf(\tcc*[f]{add estimated size of partition to total size}){$\text{\RICE}_{p} > 0$}{
    $\text{\PSIZE} \leftarrow \text{\PSIZE} + 4 + ((1 + \text{\RICE}_{p}) \times \text{\PLEN}_{p}) + \left\lfloor\frac{\text{\PSUM}_{p}}{2 ^ {\text{\RICE}_{p} - 1}}\right\rfloor - \left\lfloor\frac{\text{\PLEN}_{p}}{2}\right\rfloor$\;
  }{
    $\text{\PSIZE} \leftarrow \text{\PSIZE} + 4 + \text{\PLEN}_{p} + (\text{\PSUM}_{p} \times 2) - \left\lfloor\frac{\text{\PLEN}_{p}}{2}\right\rfloor$\;
  }
}
\BlankLine
\Return $\left\lbrace\begin{tabular}{l}
$\text{\RICE}$ \\
$\text{\PARTITION}$ \\
$\text{\PSIZE}$ \\
\end{tabular}\right.$\;
\EALGORITHM
}

\begin{figure}[h]
\includegraphics{flac/figures/residual.pdf}
\end{figure}

\clearpage

\subsubsection{Residual Encoding Example}
Given a set of residuals \texttt{[2, 6, -2, 0, -1, -2, 3, -1, -3]},
block size of 10 and predictor order of 1:
{\relsize{-1}
  \begin{align*}
  \intertext{$\text{partition order}~o = 0$:}
  \textsf{plength}_{0~0} &\leftarrow 10 \div 2 ^ 0 - 1 = 9 \\
  \textsf{partition}_{0~0} &\leftarrow \texttt{[2, 6, -2, 0, -1, -2, 3, -1, -3]} \\
  \textsf{partition sum}_{0~0} &\leftarrow 2 + 6 + 2 + 0 + 1 + 2 + 3 + 1 + 3 = 20 \\
  \textsf{Rice}_{0~0} &\leftarrow \textbf{1}~~(9 \times 2 ^ \textbf{1} < 20 \text{ and } 9 \times 2 ^ \textbf{2} > 20) \\
  \textsf{partitions size}_0 &\leftarrow 0 + 4 + ((1 + 1) \times 9) + \left\lfloor\frac{20}{2 ^ 1 - 1}\right\rfloor - \left\lfloor\frac{9}{2}\right\rfloor = \textbf{38} \\
  \intertext{$\text{partition order}~o = 1$:}
  \textsf{plength}_{1~0} &\leftarrow 10 \div 2 ^ 1 - 1 = 4 \\
  \textsf{partition}_{1~0} &\leftarrow \texttt{[2, 6, -2, 0]} \\
  \textsf{partition sum}_{1~0} &\leftarrow 2 + 6 + 2 + 0 = 10 \\
  \textsf{Rice}_{1~0} &\leftarrow \textbf{1}~~(4 \times 2 ^ \textbf{1} < 10 \text{ and } 4 \times 2 ^ \textbf{2} > 10) \\
  \textsf{partitions size}_1 &\leftarrow 0 + 4 + ((1 + 1) \times 4) + \left\lfloor\frac{10}{2 ^ 1 - 1}\right\rfloor - \left\lfloor\frac{4}{2}\right\rfloor = \textbf{20} \\
  \textsf{plength}_{1~1} &\leftarrow 10 \div 2 ^ 1 = 5 \\
  \textsf{partition}_{1~1} &\leftarrow \texttt{[-1, -2, 3, -1, -3]} \\
  \textsf{partition sum}_{1~1} &\leftarrow 1 + 2 + 3 + 1 + 3 = 10 \\
  \textsf{Rice}_{1~1} &\leftarrow \textbf{0}~~(5 \times 2 ^ \textbf{0} < 10 \text{ and } 5 \times 2 ^ \textbf{1} = 10) \\
  \textsf{partitions size}_1 &\leftarrow \textbf{20} + 4 + 5 + (10 \times
  2) - \left\lfloor\frac{5}{2}\right\rfloor = \textbf{47}
\end{align*}}
\par
\noindent
Since block size of $10 \bmod 2 ^ 2 \neq 0$, we stop at partition order 1
because the list of residuals can't be divided equally into more partitions.
And because $\textsf{partitions size}_0$ of 38 is smaller than
$\textsf{partitions size}_1$ of 47, we use partition order 0
to encode our residuals into a single partition with 9 residuals.

\begin{figure}[h]
  \includegraphics{flac/figures/residuals-enc-example.pdf}
\end{figure}
